\textbf{Cân bằng bức xạ nhiệt của Trái Đất và hiệu ứng nhà kính} 
        \begin{enumerate}[label=\textbf{\arabic*,}]\itemsep0em 
        \item \textbf{Trái Đất khi không có khí quyển} \\
        Coi rằng Trái Đất là một vật đen tuyệt đối, và giả sử nhiệt độ trên bề mặt Trái Đất được phân bố đều. Hãy ước tính nhiệt độ \(T_{0}\) này, với giả sử rằng công suất bức xạ của Mặt Trời là \(L = 3.85 \times 10^{26}\ \, \si{W}\) và khoảng cách từ Trái Đất đến mặt trời là \(d_{S-E} = 1.50 \times 10^{11} \, \si{m} \), hằng số Stefan Boltzmann \( \sigma = 5.67 \times 10^{-8} \si{W/m^2K^4}\). 
        \item \textbf{Mô hình đơn giản nhất của hiệu ứng nhà kính} \\
        Mô hình này gồm 2 phần:
        \begin{itemize}
            \item Một lớp khí quyển duy nhất cho truyền qua hoàn toàn bức xạ của Mặt Trời mà chỉ hấp thụ bức xạ do Trái Đất phát ra với hệ số \(\varepsilon_{a} = 0.77\). Nguyên nhân của sự khác nhau của hệ số hấp thụ này là do bức xạ của bề mặt Trái Đất mạnh nhất ở dải hồng ngoại, nơi mà khí quyển hấp thụ gần như toàn bộ ánh sáng, trong khi đó khí quyển lại gần như không hấp thụ ánh sáng trong vùng khả kiến mà Mặt Trời phát xạ mạnh nhất. Biểu đồ mức độ hấp thụ ánh sáng ở các bước sóng khác nhau của khí quyển và lời giải thích có thể tìm thấy ở \href{https://www.aos.wisc.edu/~aos121br/radn/radn/sld009.htm#}{đây}.
            \item Bề mặt Trái Đất phản xạ bức xạ chiếu tới từ Mặt Trời với hệ số \(\alpha = 0.28 \) và hấp thụ hoàn toàn bức xạ của khí quyển phát ra.
        \end{itemize}
        Nhiệt độ trên bề mặt và của lớp khí quyển đó được phân bố đều. Công suất phát xạ của bề mặt Trái Đất và của lớp khí quyển tuân theo công thức \(P = \sigma \varepsilon A T^{4}\), trong đó \(\sigma\)  là hằng số Stefan - Boltzmann, \(A\) là diện tích bề mặt phát xạ và T là nhiệt độ bề mặt phát xạ. Đối với bề mặt Trái Đất, \(\varepsilon = 1\) (phát xạ như vật đen tuyệt đối), còn đối với lớp khí quyển, \(\varepsilon = \varepsilon_{a}\) (tức là hệ số hấp thụ bức xạ từ Trái Đất của lớp khí quyển bằng với hệ số phát xạ nhiệt ra khỏi lớp khí quyển). Tính nhiệt độ bề mặt Trái Đất \(T_s\) và nhiệt độ bề mặt lớp khí quyển \(T_a\) khi đạt trạng thái cân bằng nhiệt. 
        \item \textbf{Khí nhà kính ảnh hưởng như thế nào tới nhiệt độ?} \\
        Giả sử do các hoạt động phát thải của con người làm tăng \(\mathrm{CO}_2\) trong khí quyển mà hệ số \(\varepsilon_{a}\) tăng một khoảng \(\Delta \varepsilon_{a}\) nhỏ. Khi đó độ tăng nhiệt độ bề mặt Trái Đất khi đạt cân bằng nhiệt \(\Delta T_{s}\) tuân theo mối quan hệ \(\Delta T_{s} = k \Delta \varepsilon_{a}\). Biểu diễn \(k\) theo \(T_{s}, \varepsilon_{a}\).
        \end{enumerate}

\begin{flushright}
    (Biên soạn bởi manhducnmd)
\end{flushright}