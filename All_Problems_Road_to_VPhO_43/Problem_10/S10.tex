\begin{enumerate}[label=\textbf{\arabic*,}] 
        \item
        Vi phân diện tích quét của đường nối giữa Trái Đất và Mặt Trời:
        \begin{equation} \label{eq1_Earth_eccentricity}
            dS = \frac{1}{2} r^2 d\phi \Rightarrow\frac{dS}{dt} = \frac{1}{2} r^2 \Dot{\phi}.
        \end{equation}
        Moment động lượng của Trái Đất quanh Mặt trời:
        \begin{equation} \label{eq2_Earth_eccentricity}
            L_E = m_E r v_\phi = m_E r^2 \Dot{\phi} .
        \end{equation}
        Từ đó ta suy ra:
        \begin{equation} \label{eq3_Earth_eccentricity}
            \frac{dS}{dt} = \frac{L_E}{2m_E}.
        \end{equation}
        Do lực hấp dẫn là lực xuyên tâm, momen động lượng của Trái Đất là hằng số, và đại lượng \(dS/dt\) không đổi. Từ đây ta có thể suy ra định luật 2 Kepler.
        \item 
        \begin{enumerate}[label=\textbf{\alph*,}]\itemsep0em
        \item Từ định luật 2 Kepler, ta suy được: 
        \begin{equation} \label{eq4_Earth_eccentricity}
            \frac{S_{1 \rightarrow 2}}{t_{1 \rightarrow 2}} = \frac{\pi ab}{T},
        \end{equation}
        trong đó, $b=a \sqrt{1-\varepsilon^2}$ là bán trục nhỏ của quỹ đạo ellipse. Phần diện tích trái đất quét qua khi đi từ 1 đến 2
        \begin{equation} \label{eq5_Earth_eccentricity}
            S_{1 \rightarrow 2} = \int_{\phi_1}^{\phi_2} \frac{1}{2} r^2 d\phi = \int_{\phi_1}^{\phi_2} \frac{1}{2} \left[ \dfrac{a \left( 1 - \varepsilon^2 \right)^2}{1 - \varepsilon \cos \phi} \right]^2 d \phi.
        \end{equation}
        Thay trở lại biểu thức trên, ta tìm được hằng số $A$ như sau
        \begin{equation} \label{eq6_Earth_eccentricity}
            A = \frac{T (1 - \varepsilon^2)^{\frac{3}{2}}}{2 \pi}.
        \end{equation}
        \item Với $\varepsilon \ll 1$, ta khai triển nhỏ số hạng chứa $\varepsilon$:
        \begin{equation} \label{eq7_Earth_eccentricity}
            \dfrac{1}{\left( 1 - \varepsilon \cos \phi \right)^2} \approx 1 + 2 \varepsilon \cos \phi.
        \end{equation}
        Thay vào biểu thức thời gian di chuyển giữa hai vị trí của trái đất và lấy tích phân, ta thu được
        \begin{equation} \label{eq8_Earth_eccentricity}
            t_{1 \rightarrow 2} \approx A[(\phi_2 - \phi_1) + 2\varepsilon(\sin{\phi_2} - \sin{\phi_1})].
        \end{equation}
        \item Do mỗi sự kiện nối tiếp nhau có góc lệch khi ngắm từ phía trái đất gần đúng là $\pi/2$ nên
        \begin{align}
        & t_{V E \rightarrow S S} \sim A\left[\pi / 2+2 \varepsilon\left(\cos \phi_{V E}-\sin \phi_{V E}\right)\right], \\
        & t_{S S \rightarrow A E} \sim A\left[\pi / 2-2 \varepsilon\left(\sin \phi_{V E}+\cos \phi_{V E}\right)\right], \\
        & t_{A E \rightarrow W S} \sim A\left[\pi / 2-2 \varepsilon\left(\cos \phi_{V E}-\sin \phi_{V E}\right)\right], \\
        & t_{W S \rightarrow V E} \sim A\left[\pi / 2+2 \varepsilon\left(\sin \phi_{V E}+\cos \phi_{V E}\right)\right] .
        \end{align}
        Từ 4 phương trình trên, ta tìm được góc $\phi_{VE}$
        \begin{equation} \label{eq13_Earth_eccentricity}
            \tan \phi_{V E}=\frac{\left(1-\tau_{1}\right)}{\left(1+\tau_{1}\right)},
        \end{equation}
        trong đó
        \begin{equation} \label{eq14_Earth_eccentricity}
            \tau_{1}=\left(\frac{t_{V E \rightarrow S S}-t_{A E \rightarrow W S}}{t_{W S \rightarrow V E}-t_{S S \rightarrow A E}}\right)=\left(\frac{\cos \phi_{V E}-\sin \phi_{V E}}{\cos \phi_{V E}+\sin \phi_{V E}}\right).
        \end{equation}
        Từ bảng số liệu, $\tau_1 \approx -3/4$ và $\varepsilon \approx 4.57 \si{Rad}$.
        Lấy
        \begin{equation} \label{eq15_Earth_eccentricity}
            \tau_{2}=\frac{t_{V E \rightarrow S S}}{t_{A E \rightarrow W S}} = \dfrac{\pi / 2+2 \varepsilon\left(\cos \phi_{V E}-\sin \phi_{V E}\right)}{\pi / 2-2 \varepsilon\left(\cos \phi_{V E}-\sin \phi_{V E}\right)}.
        \end{equation}
        Ta tìm được
        \begin{equation} \label{eq16_Earth_eccentricity}
            \varepsilon \sim \frac{\pi\left(\tau_{2}-1\right)}{4\left(1+\tau_{2}\right)\left(\cos \phi_{V E}-\sin \phi_{V E}\right)} \approx 0.0152.
        \end{equation}
        % (Tham khảo bạn Quân Bảo) 
        Từ 4 phương trình trên, ta thu được hai phương trình: 
        \begin{align}
            & \frac{t_{V E \rightarrow S S}}{t_{A E \rightarrow W S}} = \frac{A\left[\pi / 2+2 \varepsilon\left(\cos \phi_{V E}-\sin \phi_{V E}\right)\right]}{A\left[\pi / 2-2 \varepsilon\left(\cos \phi_{V E}-\sin \phi_{V E}\right)\right]}, \\
            & \frac{t_{W S \rightarrow V E}}{t_{S S \rightarrow A E}} = \frac{A\left[\pi / 2+2 \varepsilon\left(\cos \phi_{V E}+\sin \phi_{V E}\right)\right]}{A\left[\pi / 2-2 \varepsilon\left(\cos \phi_{V E}+\sin \phi_{V E}\right)\right]}. \\
        \end{align}
        Chuyển vế và rút gọn, ta thu được:
        \begin{align}
             & \varepsilon\left(\cos \phi_{V E}-\sin \phi_{V E}\right) =  \frac{\pi \left(t_{V E \rightarrow S S} -t_{A E \rightarrow W S}\right)}{4\left(t_{V E \rightarrow S S} + t_{A E \rightarrow W S}\right)} = a ,\\
             & \varepsilon\left(\cos \phi_{V E}+\sin \phi_{V E}\right) =  \frac{\pi \left(t_{W S \rightarrow V E} -t_{S S \rightarrow A E}\right)}{4\left(t_{W S \rightarrow V E} + t_{S S \rightarrow A E}\right)} = b.
        \end{align}
        Ta rút ra được $\varepsilon \cos \phi_{V E} = (a+b)/2, \varepsilon \sin \phi_{V E} = (a-b)/2$. Cuối cùng,
        \begin{equation} \label{eq22_Earth_eccentricity}
            \varepsilon = \left(\left(\frac{a+b}{2}\right)^2 + \left(\frac{a-b}{2}\right)^2 \right)^{1/2} \approx 0.0166
        \end{equation}
    \end{enumerate}
\end{enumerate}
\textit{Theo số liệu của USNO, tâm sai của trái đất là $\varepsilon = 0.0167$, tức là phép đánh giá thô của chúng ta đã có những hiệu quả nhất định trong việc thiết lập mô hình tính toán tâm sai của trái đất.}

\textit{Lưu ý: Việc khai triển làm xuất hiện các số hạng $\varepsilon$ bậc cao hơn 1 có thể gây ra sai số khá lớn và các đáp án đó sẽ không được chấp nhận.}

\textbf{Biểu điểm}
\begin{center}
\begin{tabular}{|>{\centering\arraybackslash}m{1cm}|>{\raggedright\arraybackslash}m{14cm}| >{\centering\arraybackslash}m{1cm}|}
    \hline
    \textbf{Phần} & \textbf{Nội dung} & \textbf{Điểm} \\
    \hline
    \textbf{1} & Viết biểu thức diện tích quét (\ref{eq1_Earth_eccentricity}) & $0.25$ \\
    \cline{2-3}
    & Viết phương trình bảo toàn momen động lượng (\ref{eq2_Earth_eccentricity}) & $0.50$ \\
    \cline{2-3}
    & Suy ra định luật 2 Kepler (\ref{eq3_Earth_eccentricity}) & $0.25$ \\
    \hline
    \textbf{2a} & Áp dụng định luật 2 Kepler (\ref{eq4_Earth_eccentricity}) & $0.25$ \\
    \cline{2-3}
    & Viết biểu thức diện tích quét & $0.50$ \\
    \cline{2-3}
    & Đối chiếu kết quả và tìm $a$ & $0.25$ \\
    \hline
    \textbf{2b} & Khai triển nhỏ số hạng $\varepsilon$ bậc 1 (\ref{eq7_Earth_eccentricity}) & $0.50$ \\
    \cline{2-3}
    & Lấy tích phân và thu được biểu thức thời gian giữa các sự kiện (\ref{eq8_Earth_eccentricity}) & $0.50$ \\
    \hline
    \textbf{2c} & Từ bảng số liệu xác định $\phi_{VE}$ (\ref{eq13_Earth_eccentricity}) & $0.50$ \\
    \cline{2-3}
    & Tính được giá trị $\varepsilon$ với một phương pháp đúng sai lệch dưới $20\%$ (\ref{eq16_Earth_eccentricity}) & $0.50$ \\
    \hline
\end{tabular}
\end{center}

\bibliographystyle{plain}
\begin{thebibliography}{}
\bibitem{10.1119/5.0127980} B. Cameron Reed. Eccentricity and orientation of Earth’s orbit from equinox and solstice times. \textit{American Journal of Physics}, 91(4):324–326, 04 2023.
\end{thebibliography}