1. Ta có thể vẽ lại mạch tương đương như sau
\begin{center}
\begin{circuitikz}[american]\draw
    (0,-3) node[ground] {} to [R,l_={$R_\text{in}$}] ++(0,3) to [R,l_={$R_1$}] ++(-3,0) to [V,l_={$v_s$}] ++(0,-3) to [short,-o]++(10,0)
    (-0.5,0) node[above right]{$\text{I}$} to [short]++ (0,1) to [R,l={$R_f$},right]++(6.5,0) to [short]++(0,-1) to [R,l_={$R_\text{out}$}] (3.5,0)
    (6,0) node[above right]{$\text{O}$} to [short,-o]++(1,0)  node[right,yshift=-1.6cm]{$v_\text{O}$}
    (3.5,0) to [cV,l=$A(v_2-v_1)$] ++ (0,-3)
    
;\end{circuitikz}
\end{center}
Tại nút I, theo KCL ta có
\begin{equation}
    \dfrac{v_s-v_\text{I}}{R_1}= \dfrac{v_\text{I}}{R_\text{in}}+\dfrac{v_\text{I}-v_\text{O}}{R_f} .
    \label{kcli}
\end{equation}
Tại nút O, theo KCL ta có
\begin{equation}
    \dfrac{v_\text{I}-v_\text{O}}{R_f}=\dfrac{v_\text{O}-A(v_2-v_1)}{R_\text{out}}
    \label{kclo} .
\end{equation}
Do đầu vào dương của OPAMP nối đất nên $v_2=0$.
Giải hệ phương trình \eqref{kcli} và \eqref{kclo} ta được 

\begin{equation}
    \dfrac{v_\text{O}}{v_s} = -\dfrac{AR_fR_\text{in}-R_\text{in}R_\text{out}  }{R_1R_f+R_1R_\text{in}+AR_1R_\text{in}+R_fR_\text{in}+R_1R_\text{out}+R_\text{in}R_\text{out}}
    \approx -2 .
\end{equation}
2. Áp dụng các thông số của OPAMP lý tưởng $R_\text{in} \to \infty$, $R_\text{out}\to 0$ và $A\to \infty$, ta được
\begin{equation}
    \dfrac{v_\text{o}}{v_s}=-\dfrac{R_f}{R_1} .
\end{equation}

3. Phân tích mạch tương tự như ý 1, ta được
\begin{equation}
    v_\text{sum}=-\left(\dfrac{R_f}{R_1}v_1+\dfrac{R_f}{R_2}v_2+\dfrac{R_f}{R_3}v_3\right) ,
\end{equation}
\begin{equation}
    v_\text{int}=-\dfrac{1}{RC}\int  v_s dt .
\end{equation}
\begin{equation}
    v_\text{ni}= \left( 1+\dfrac{R_f}{R} \right) v_s
\end{equation}
4. Bằng cách kết hợp 2 mạch tích phân và 1 mạch đảo. Ta được một cách mắc mạch như sau

\begin{center}
\begin{circuitikz}[american,scale=0.8,font=\footnotesize]
\ctikzset{bipoles/length=11mm}\draw
(0,0) node[op amp] (opamp1) {}
%Positive input connect to ground
 (opamp1.+) to [short] ++ (-0.5,0) to [short] ++(0,-0.5) node[ground]{} 
 %Negative input 1 connects with external source
 (opamp1.-) to [short] ++(-6,0) to[R,l_={$R_1$}] ++(0,-2)to [V, l_=$-d(\si{V})$] ++(0,-2) to node[ground]{} ++(0,0)
 %Negative input 3 connects with d/dt v_0
 (opamp1.-) to [short] ++(-3,0) to[R,l_={$R_3$}] ++(0,-3)to  [short] ++(5.35,0) to (opamp1.out)
 %Negative input 2 connects with v_0
(opamp1.-) to [short] ++(-4.5,0) to[R,l_={$R_2$}] ++(0,-6)to  [short] ++(15.3,0) 
 
 %out=d\dt v_0
 (opamp1.out) to [short] ++(0.62,0) node[above] {$\dfrac{dx}{dt}$}
 %negative feedback
 (opamp1.-) to [short] ++(0,1) to [C,l={$C_1$}] ++(2.37,0) to (opamp1.out)
 %initial condition for v'_0
 (opamp1.-) to [short] ++(0,3) to [V,invert,l=e\si{(V)}]++(1.2,0) to [opening switch,l={}] ++(1.2,0) to (opamp1.out)
 %Second integrator
 (6,-0.5) node[op amp] (opamp2) {}
 (opamp1.out) to [R,l=$R_4$] (opamp2.-)
 %positive input to ground
 (opamp2.+) to [short] ++ (-0.5,0) to [short] ++(0,-0.5) node[ground]{}
 %
 (opamp2.-) to [short] ++(0,1) to [C,l={$C_2$}] ++(2.37,0) to (opamp2.out)
 %
 (opamp2.-) to [short] ++(0,3) to [V,l=f\si{(V)}]++(1.2,0) to [opening switch,l={}] ++(1.2,0) to (opamp2.out)
 %
 (8.35,-4) node[op amp] (opamp3){}
 (opamp2.out) to [R,l=$R_5$] (opamp3.-)
 (opamp2.out) node[above right]{$-x$} 
 %
 (opamp2.out) to [short]++(2.4,0) to [R ,l={$R_6$}] (opamp3.out) to [short,-*]++(0.3,0) node[right]{$x$}
 (opamp3.out) to [short] ++(0,-1.5)
 (opamp3.+) to ++(-0.5,0) node[ground]{}
 
;\end{circuitikz}
\end{center}
Giải thích mạch

Xét phương trình vi phân ban đầu
\begin{equation}
    \begin{split}
        &a\dfrac{d^2x}{dt^2}+b\dfrac{dx}{dt}+cx=d\\
        &\Longrightarrow \dfrac{d^2x}{dt^2}=\dfrac{1}{a}\left(d-cx-b\dfrac{dx}{dt}\right)\\
        &\Longrightarrow \dfrac{dx}{dt}=- \int_0^{t}\dfrac{1}{a}\left(-d+cx+b\dfrac{dx}{dt}\right) + \dot{x}(0) .
    \end{split}
\end{equation}

Từ phương trình trên, để lấy được điện áp $\dot{x}(t)$, ta cần một mạch tích phân với đầu vào điện thế $-d$, điện thế $x(t)$ và điện thế $\dot{x}(t)$, cùng với một điện thế $\dot{x}(0)=e$ đặt vào đầu ra của OPAMP thứ nhất.

Tiếp theo, để có được $x(t)$

\begin{equation}
    -x(t)=+\int_0^t \dot{x}(t)dt - x(0) .
\end{equation}

Do đó ta dùng một mạch tích phân nữa với đầu vào là điện thế $\dot{x}(t)$, ở đầu ra của mạch này, ta được điện thế $-x(t)$. Do $x(0)=f(\si{V})$ nên ta cần đặt một điện thế có giá trị $-f(\si{V})$ ở đầu ra của OPAMP thứ 2.

Cuối cùng, ta dùng một mạch đảo để đổi dấu điện áp đầu vào $-x(t)$.

Từ các phương trình, ta cần phải chọn các linh kiện thoả mãn
\begin{equation}
        R_1C_1=a, R_2C_1=\dfrac{a}{c}, R_3C_1=\dfrac{a}{b}, R_4C_2=1, R_6=R_5 .
\end{equation}
5. Đây là một cách mắc mạch phổ biến để tạo ra một "điện trở âm":
\begin{center}
\begin{circuitikz}[american]\draw
(0,0) node[op amp] (opamp) {}
 (opamp.+) to [short] ++ (-1.5,0) to [V,l=$v_\text{in}$,i<=$i_\text{in}$] ++(0,-2) node[ground]{} 
 (opamp.-) to [R,l_={$R_1$},i>={$i_1$}] ++(-3,0) to [short] ++(0,-3) to [short]++(1.5,0) 
 (opamp.-) to [short] ++(0,1) to [R,l={$R_f$},i<=$i_f$] ++(2.4,0) to  (opamp.out) node[right]{O}
 (opamp.-) node[above left]{N}
 (opamp.+) node[above]{I}
 (opamp.+) to [short] ++(0,-1) to [R,l=$R_2$,i>={$i_2$}] ++(2.39,0) to [short] (opamp.out)
 ;\end{circuitikz}
\end{center}
Do OPAMP lý tưởng nên $v_\text{I}=v_\text{N}=v_\text{in}$.

Tại N ta có
\begin{equation}
    i_f=i_1 \Longrightarrow \dfrac{v_\text{O}-v_\text{in}}{R_f}=\dfrac{v_\text{in}}{R_1}\Longrightarrow v_\text{O}=v_\text{in} (1+\dfrac{R_f}{R_1}) .
\end{equation}

Tại I ta có:
\begin{equation}
    i_\text{in}=i_2=\dfrac{v_\text{in}-v_\text{O}}{R_2} .
\end{equation}

Từ 2 phương trình trên ta được
\begin{equation}
    R_\text{in}=\dfrac{v_\text{in}}{i_\text{in}}=-\dfrac{R_1R_2}{R_f} .
\end{equation}
\textbf{Biểu điểm}
\begin{center}
\begin{tabular}{|>{\centering\arraybackslash}m{1cm}|>{\raggedright\arraybackslash}m{14cm}| >{\centering\arraybackslash}m{1cm}|}
    \hline
    \textbf{Phần} & \textbf{Nội dung} & \textbf{Điểm} \\
    \hline
    \textbf{1} & Vẽ được mạch tương đương & $0.50$ \\
    \cline{2-3}
    & Viết các phương trình Krichhoff và giải ra được độ lợi & $0.25$ \\
    \hline
    \textbf{2} & Dẫn ra được công thức độ lợi cho OPAMP lý tưởng & $0.25$ \\
    \hline
    \textbf{3} & Giải được mạch khuếch đại không đảo & $0.25$ \\
    \cline{2-3}
    & Giải được mạch tích phân & $0.50$ \\
    \cline{2-3}
    & Giải được mạch cộng & $0.25$ \\
    \hline
    \textbf{4} & Vẽ được mạch để giải phương trình vi phân & $0.50$ \\
    \cline{2-3}
    & Giải thích và biện luận đúng các tham số của mạch  & $0.50$ \\
    \hline
    \textbf{5}
    & Vẽ được mạch điện trở âm  & $0.50$ \\
    \cline{2-3}
    & Biện luận và tính giá trị của điện trở & $0.50$ \\ 
    
    \hline
\end{tabular}
\end{center}
\bibliographystyle{plain}
\begin{thebibliography}{}
\bibitem[]{YKLim} Yong-Kyu Lim, Problems and Solutions on Electromagnetism 
\bibitem[]{floyd1992electronic} T.L. Floyd. Electronic Devices. Merrill’s international series in electrical and electronics technology.
Merrill, 1992.
\end{thebibliography}