\textbf{Ellipsoid dẫn}

\begin{enumerate}
    \item Một dây thẳng tích điện đều có điện tích $Q$ dài $2L$. Chọn hệ tọa độ trụ như hình 1.
    \begin{enumerate}[label=\textbf{\alph*,}]\itemsep0em
    \item Tính điện thế dây gây ra tại một điểm bất kỳ trong không gian không nằm trên sợi dây.
    \item Viết phương trình mặt đẳng thế có điện thế $\phi$ gây ra bởi dây.
    \end{enumerate}
    \item Một vật dẫn hình khối tạo bởi việc quay hình ellipse có bán trục lớn là $a$ và bán trục nhỏ là $b$ quanh trục chứa bán trục lớn của nó (hình 2).
    \begin{enumerate}[label=\textbf{\alph*,}]\itemsep0em
    \item Hãy tìm điện dung của khối ellipsoid tròn xoay này.
    \item Tìm mật độ điện mặt $\sigma$ của khối ellipsoid tròn xoay khi nó được tích điện tích $Q$.
    \end{enumerate}
\end{enumerate}

\begin{center}
\begin{minipage}{0.4\textwidth}
\begin{tikzpicture}[scale=1.2]
    \draw[fill=lightgray] (-2,-0.03) rectangle (2,0.03);
    \draw[-Stealth] (0,0) to (3,0);
    \draw[-Stealth] (0,0) to (0,2);
    \filldraw[color=black, fill=black, ultra thick] (0,0) circle (0.05);
    \draw (-0.2,0) node[below]{$O$} (0,1.8) node[left]{$\rho$} (2.8,0) node[below]{$z$};
    \draw[dashed] 
    (2,0) to (2,-0.5)
    (-2,0) to (-2,-0.5);
    \draw[Stealth-Stealth] (-2,-0.5) to (2,-0.5);
    \draw (0,-0.5) node[below]{$2L$} (-1.5,0) node[above]{$Q$};
    \filldraw[color=black, fill=black, ultra thick] (1,1.6) circle (0.05);
    \draw (1,1.6) node[right]{$(\rho,z)$};
    \draw (0,-2.2) node{\textbf{(a)}};
\end{tikzpicture}
\end{minipage}
\begin{minipage}{0.4\textwidth}
\begin{tikzpicture}[scale=1.2]
    \draw[shading=axis, left color=darkgray, right color=lightgray!80, shading angle=135, anchor=north] (0,0) ellipse (2.5 and 1);
    \draw[-Stealth] (0,0) to (3,0);
    \draw[-Stealth] (0,0) to (0,2);
    \filldraw[color=black, fill=black, ultra thick] (0,0) circle (0.05);
    \draw (-0.2,0) node[below]{$O$} (0,1.8) node[left]{$\rho$} (2.8,0) node[below]{$z$};
    \draw[dashed] 
    (-2.5,0) to (-2.5,-1.5)
    (2.5,0) to (2.5,-1.5)
    (0,1) to (-3,1)
    (0,-1) to (-3,-1);
    \draw[Stealth-Stealth] (-2.5,-1.5) to (2.5,-1.5);
    \draw[Stealth-Stealth] (-3,1) to (-3,-1);
    \draw (-3,0) node[left]{$2b$} (0,-1.5) node[below]{$2a$};
    \draw (0,-2.2) node{\textbf{(b)}};
\end{tikzpicture}
\end{minipage} \\
Hình 1: \textbf{(a)} Thanh tích điện đều trong hệ tọa độ trụ. \textbf{(b)} Vật dẫn ellipsoid.
\end{center}

\begin{flushright}
    (Biên soạn bởi Log và Pềct)
\end{flushright}