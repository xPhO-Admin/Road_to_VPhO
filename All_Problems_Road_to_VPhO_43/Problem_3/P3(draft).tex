\begin{enumerate}
    \item Xét hai hệ điện tích $A$ và $B$. Hệ $A$ gồm $n$ điện tích điểm $q_{1a}$, $q_{2a}$, $q_{3a}$,..., $q_{na}$ lần lượt nằm tại các vị trí  $\Vec{r}_{1a}$, $\Vec{r}_{2a}$,  $\Vec{r}_{3a}$,..., $\Vec{r}_{na}$. Hệ $B$ gồm $m$ điện tích điểm $q_{1b}$, $q_{2b}$, $q_{3b}$,..., $q_{nb}$ lần lượt nằm tại các vị trí  $\Vec{r}_{1b}$, $\Vec{r}_{2b}$,  $\Vec{r}_{3b}$,..., $\Vec{r}_{mb}$. Gọi $\phi_A (\Vec{r})$ và $\phi_B (\Vec{r})$ là lần lượt là điện thế do hệ $A$ và hệ $B$ gây lên một điểm nằm tại $\Vec{r}$. Chứng minh rằng định lý giao hoán Green (Green's reciprocity theorem):
    $$ \sum_{i=1}^m \phi_A (\Vec{r}_{ib}) q_{ib} = \sum_{j=1}^n \phi_B (\Vec{r}_{ja}) q_{ja}.$$
    \item Chứng minh mặt đẳng thế có điện thế $V$ gây ra bởi một dây tích điện đều dài $2L$ có điện tích $Q$ là một hình ellipse tròn xoay. Từ đó, hãy xác định điện dung của một vật dẫn hình ellipse tròn xoay.
    \item Một vật dẫn hình khối tròn xoay được tạo ra khi ellipse, có bán trục lớn là $a$ và bán trục nhỏ là $b$, quay quanh trục chứa bán trục lớn của nó. Đặt vật dẫn hình ellipse tròn xoay này cách một vỏ cầu kim loại mỏng có bán kính $R$ một khoảng $r$ rất lớn so với kích thước hai vật. Cấp vào ellipse tròn xoay một điện thế $\phi_1$ và cấp vào vỏ cầu điện thế $\phi_2$. Hãy tìm điện tích $Q_1$ trên khối ellipse tròn xoay và điện tích $Q_2$ trên vỏ cầu.
\end{enumerate}

\textit{Ghi chú: trong toàn bộ bài toán này, ta chọn mốc điện thế bằng $0$ ở xa vô cùng.}

\begin{flushright}
    (Biên soạn bởi Pềct và Log)
\end{flushright}