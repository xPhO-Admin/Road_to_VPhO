%Bài toán đề xuất bởi NVPL

\textbf{Pha trộn và dao động neutrino}

\textit{Neutrino} là một trong các hạt cơ bản của vũ trụ mà đã được con người phát hiện ra. Các nghiên cứu về neutrino là một trong những nghiên cứu tiền tuyến của vật lý hạt cơ bản do các hạt neutrino chuyển động rất nhanh (tiệm cận vận tốc ánh sáng) và chỉ tham gia vào tương tác yếu và tương tác hấp dẫn nên rất khó để "bắt" được dấu vết của các hạt neutrino. Theo \textit{Mô hình Chuẩn (Standard Model)}, các hạt neutrino được giả định là không có khối lượng; tuy nhiên số liệu từ thí nghiệm phân rã hạt $\beta$ của tritium $^3\text{H}$ tương thích với khả năng các hạt neutrino có khối lượng khác không. Trong bài tập này ta sẽ tìm hiểu về hiện tượng dao động neutrino dựa trên các cách tiếp cận phổ thông và trực quan hơn bên cạnh cách tiếp cận chính thống là lý thuyết trường lượng tử để đánh giá và tìm hiểu cách mà các nhà vật lý đã kiểm chứng được các hạt neutrino có khối lượng.

\begin{enumerate}
    \item \textbf{Hiện tượng phách}\\
    Theo cơ học lượng tử, hàm sóng liên kết của hai trạng thái có tần số gần giống nhau sẽ xảy ra hiện tượng phách, tương tự như khi chồng chập hai sóng âm có tần số gần giống nhau trong vật lý cổ điển. Giả sử ta có hai hàm sóng $A_1$ và $A_2$ có tần số góc khác nhau lần lượt là $\omega_1$ và $\omega_2$:
    \begin{align*}
        A_1(t)&=Ae^{-i\omega_1 t},\\
        A_2(t)&=-Ae^{-i\omega_2 t}.
    \end{align*}
    \begin{enumerate}
    \item Xác định biểu thức môđun phức bình phương của hàm sóng liên kết hai trạng thái tại thời điểm $t$ bất kỳ.
    \item Giả sử $\omega_1$ và $\omega_2$ là tần số góc ứng với hai trạng thái của hạt nào đó chuyển động với vận tốc tiệm cận vận tốc ánh sáng có năng lượng lần lượt là $E_1$ và $E_2$. Dựa trên quan điểm của de Broglie về lưỡng tính sóng-hạt của vật chất. Biểu diễn biểu thức môđun phức bình phương của hàm sóng liên kết hai trạng thái tại thời điểm $t$ bất kỳ theo năng lượng $E_1$, $E_2$ và các hằng số liên quan.
    \end{enumerate}
    \item \textbf{Độ dài dao động của neutrino}\\
    Các nhà vật lý đã giả thiết rằng, nếu hạt neutrino $a$ nào đó thực sự có khối lượng, ứng với mỗi trạng thái năng lượng $E_n$ của nó thì hạt sẽ có khối lượng $m_n$ tương ứng. Nếu xảy ra sự liên kết giữa hai hàm sóng mô tả hai trạng thái có khối lượng khác nhau của hạt neutrino thì sẽ xảy ra hiện tượng phách, sự liên kết này được gọi là \textit{pha trộn neutrino (neutrino mixing)}. Trong quá trình pha trộn neutrino, qua thời gian, hạt neutrino $a$ liên tục biến thành hạt neutrino $b$ và ngược lại, ứng với pha của hàm sóng liên kết, quá trình này được gọi là \textit{dao động neutrino (neutrino oscillations)}. Giả sử ban đầu ở lò phản ứng hạt nhân chỉ phát ra chùm hạt neutrino $a$ theo phương $x$, tỉ lệ giữa số hạt neutrino $b$ biến đổi từ neutrino $a$ ở thời điểm $t$ so với số hạt neutrino $a$ ở thời điểm ban đầu được tính theo biểu thức:
    $$P_{\alpha \rightarrow \beta}=\sin^2(2\theta)\sin^2\left(\dfrac{E_m-E_n}{2\hbar}t\right)=\sin^2(2\theta)\sin^2\left(\dfrac{x}{L_0}\right).$$
    Trong đó $\theta$ được gọi là góc pha trộn, $L_0$ được gọi là độ dài dao động của neutrino.\\
    Xác định biểu thức độ dài dao động của neutrino theo $E_m$, $E_n$ và các hằng số liên quan.

\item \textbf{Thí nghiệm KamLAND}\\
Thí nghiệm KamLAND (The Kamioka Liquid-scintillator Anti-Neutrino Detector) nghiên cứu về dao động của phản hạt neutrino electron $\overline{\nu_e}$ sau khi bay ra từ lò phản ứng hạt nhân với vận tốc tiệm cận vận tốc ánh sáng ở khoảng cách $x \approx 170 \ \si{km}$ so với lò phản ứng. Tại đó, số hạt $\overline{\nu_e}$ chỉ còn $20\%$ so với ban đầu. Giả thiết các hạt $\overline{\nu_e}$ chỉ có các trạng thái $E_m$ và $E_n$. Năng lượng trung bình của các hạt $\overline{\nu_e}$ là $E \approx E_m \approx E_n \approx 4 \ \si{MeV}$ và $E_m \gg m_m c^2$, $E_n \gg m_n c^2$. Cho biết góc pha trộn $\theta = 45^o$. \\
Tính hiệu số bình phương khối lượng $\Delta m^2 = m_m^2 - m_n^2$ theo đơn vị $\si{(eV)^2/c^4}$.
\end{enumerate}

\begin{flushright}
    (Biên soạn bởi Nhân viên phòng lab)
\end{flushright}