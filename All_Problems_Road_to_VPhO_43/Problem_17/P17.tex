\textbf{Giới hạn GZK}

\hspace{1 cm}Vào năm 1966, Greisen, Zatsepin và Kuzmin lập luận rằng chúng ta không thể quan sát được các tia vũ trụ (các proton năng lượng cao ngoài không gian đi vào khí quyển) ở trên một ngưỡng năng lượng nào đó do sự tương tác của chúng với bức xạ vi phông vũ trụ (CMB).
\begin{enumerate}
    \item Cho rằng vũ trụ có nhiệt độ 2.73 K. Tìm năng lượng trung bình của một photon ngoài không gian. Kết quả biểu diễn theo đơn vị (eV). Giả sử rằng photon tuân theo quy luật thống kê Bose-Einstein.
    \item Giả sử proton $p^+$ tương tác với photon $\gamma$ theo phương trình:
    $$p^+ + \gamma \longrightarrow p^+ + \pi^0$$ Tìm năng lượng cần thiết của proton đến để phản ứng có thể xảy ra. Biết khối lượng của proton và pion lần lượt là $m_p =938 $ MeV và $m_{\pi}= 135$ MeV.
    \item Tìm năng lượng của proton sau phản ứng.\\
     Hiện tượng này lần đầu tiên được quan sát thực nghiệm vào năm 2008, sau hơn 40 năm phỏng đoán được đưa ra. 
     
     Cho các tính phân:
     $$\int_{0}^{\infty}\frac{x^3}{e^x-1}dx= \frac{\pi^4}{15}, \quad \int_{0}^{\infty}\frac{x^2}{e^x-1}dx= 2.40$$
     Phân bố Bose-Einstein $n(E) = \dfrac{g}{e^{E/k_BT}-1}$, trong đó $g$ là số phân cực của hạt.
\end{enumerate}

\begin{flushright}
    (Biên soạn bởi Bourbaki và Log)
\end{flushright}