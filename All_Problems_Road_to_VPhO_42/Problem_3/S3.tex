\begin{enumerate}[label=\textbf{\alph*,}]\itemsep0em
\item Theo tính đối xứng cầu của hệ, ta có thể dễ dàng suy ra mật độ dòng
\begin{equation} \label{eq1_P3_d1}
    j(r) = \frac{I}{2\pi h r}.
\end{equation}
Một hạt electron chuyển động đến vị trí có bán kính $r$ sẽ nhận được một công $eV(r)$ của điện trường, xem rằng công này được chuyển hoá làm tăng động năng của hạt điện tích $\frac{1}{2}mv^2$, từ đó ta xác định được vận tốc của hạt:
\begin{equation} \label{eq2_P3_d1}
    \frac{1}{2}mv^2 = eV(r) \Rightarrow v(r)= \sqrt{ \frac{2eV(r)}{m} }.
\end{equation}
Theo định nghĩa mật độ dòng:
\begin{equation} \label{eq3_P3_d1}
    j(r) = \rho v \Rightarrow \rho = \frac{j}{v} = \frac{I}{2\pi h r} \sqrt{ \frac{m}{2eV} } .
\end{equation}

\item Trong hệ đối xứng cầu, cường độ điện trường tại bán kính $r$ là
\begin{equation} \label{eq4_P3_d1}
    E = - \frac{dV}{dr}.
\end{equation}
Áp dụng định luật Gauss cho một vỏ trụ bán kính $r$ và độ dày $dr$, ta có
\begin{equation} \label{eq5_P3_d1}
    \begin{split}
        E(r+dr) \cdot 2 \pi h (r+dr) - E(r) \cdot 2 h \pi r &= \frac{\rho}{\varepsilon_0} \cdot 2 \pi h r dr \\
        \Rightarrow \frac{d}{dr} \left( E \cdot r \right) = \frac{\rho}{\varepsilon_0} r.
    \end{split}
\end{equation}
Từ (\ref{eq4_P2_d1}) và (\ref{eq5_P3_d1}), ta được:
\begin{equation} \label{eq6_P3_d1}
    \frac{1}{r} \frac{d}{dr} \left( r \frac{dV}{dr} \right) = \frac{\rho}{\varepsilon_0}.
\end{equation}
\item Từ (\ref{eq3_P2_d1}) và (\ref{eq6_P3_d1}) ta thu được phương trình vi phân của $V(r)$:
\begin{equation} \label{eq7_P3_d1}
    \frac{d}{dr} \left( r \frac{dV}{dr} \right) = \frac{I}{2\pi \varepsilon_0 h} \sqrt{ \frac{m}{2eV} } .
\end{equation}
Thế dạng nghiệm $V(r)=Ar^\alpha$ vào phương trình (\ref{eq7_P3_d1}), ta được:
\begin{equation} \label{eq8_P3_d1}
    A \alpha^2 r^{\alpha-1} = \frac{I}{2 \pi \varepsilon_0 h} \sqrt{ \frac{m}{2e A} } r^{-\alpha/2}.
\end{equation}
Đồng nhất hệ số 2 vế, ta được:
\begin{equation} \label{eq9_P3_d1}
    \alpha-1=-\frac{\alpha}{2} \Rightarrow \alpha = \frac{2}{3}.
\end{equation}
Và 
\begin{equation} \label{eq10_P3_d1}
    A \alpha^2 = \frac{I}{2 \pi \varepsilon_0 h} \sqrt{ \frac{m}{2e A} } \Rightarrow A= \left( \frac{9I}{8\pi \varepsilon_0 h} \sqrt{\frac{m}{2e}} \right)^{2/3}.
\end{equation}
\item Thay (\ref{eq9_P3_d1}) và (\ref{eq10_P3_d1}) vào công thức nghiệm
\begin{equation} \label{eq11_P3_d1}
    V(r) = \left( \frac{9I}{8\pi \varepsilon_0 h} \sqrt{\frac{m}{2e}} \right)^{2/3} r^{2/3}.
\end{equation}
Tại $r=R$ thì $V(R)=U$ nên
\begin{equation} \label{eq12_P3_d1}
    U = \left( \frac{9I}{8\pi \varepsilon_0 h} \sqrt{\frac{m}{2e}} \right)^{2/3} R^{2/3}.
\end{equation}
Hay
\begin{equation} \label{eq13_P3_d1}
    I = \left( \frac{8\pi \varepsilon_0 h}{9R} \sqrt{ \frac{2e}{m} }\right) U^{3/2}.
\end{equation}
Như vậy
\begin{equation} \label{eq14_P3_d1}
    K = \frac{8\pi \varepsilon_0 h}{9R} \sqrt{ \frac{2e}{m} } .
\end{equation}
\end{enumerate}

\textbf{Biểu điểm} 
\begin{center}
\begin{tabular}{|>{\centering\arraybackslash}m{1cm}|>{\raggedright\arraybackslash}m{14cm}| >{\centering\arraybackslash}m{1cm}|}
    \hline
\textbf{Phần} & \textbf{Nội dung} & \textbf{Điểm} \\
    \hline
    \textbf{a} & Tìm được mật độ dòng $j$ theo $I$ (\ref{eq1_P3_d1}) & $0.25$ \\
    \cline{2-3}
    & Tìm vận tốc $v$ theo điện thế $V$ (\ref{eq2_P3_d1}) & $0.50$ \\
    \cline{2-3}
    & Tìm mật độ điện khối $\rho$ theo $V$ và $r$ (\ref{eq3_P3_d1}) & 0.25 \\
    \hline
    \textbf{b} & Viết biểu thức điện trường theo điện thế (\ref{eq4_P3_d1}) & $0.25$ \\
    \cline{2-3}
    & Áp dụng định luật Gauss cho vỏ trụ (\ref{eq5_P3_d1}) & $0.50$ \\
    \cline{2-3}
    & Kết hợp 2 phương trình để viết được phương trình vi phân (\ref{eq6_P3_d1}) & 0.25 \\
    \hline
    \textbf{c} & Hoàn chỉnh phương trình vi phân (\ref{eq7_P3_d1}) & $0.25$ \\
    \cline{2-3}
    & Thế dạng nghiệm của phương trình (\ref{eq8_P3_d1}) & $0.25$ \\
    \cline{2-3}
    & Tính được $\alpha$ (\ref{eq9_P3_d1}) & $0.50$ \\
    \cline{2-3}
    & Tính được $A$ (\ref{eq10_P3_d1}) & $0.50$ \\ 
    \hline
    \textbf{d} & Thay điều kiện biên và viết được $U$ theo $I$ (\ref{eq12_P3_d1}) & $0.25$ \\ 
    \cline{2-3}
    & Tính hệ số $K$ (\ref{eq14_P3_d1}) & $0.25$ \\
    \hline
\end{tabular}
\end{center}

\noindent \textbf{Mở rộng vấn đề:} \\
Đây là một bài toán không có các tính toán phức tạp nhưng khá khó để nhìn được ra đường hướng giải quyết, đòi hỏi người giải cần có một kiến thức nền khá tốt. Đối với những người đã học tương đối sâu về phân bố của điện từ trường và biết sử dụng các công cụ toán tốt, không khó để nhận thấy câu hỏi phần \textbf{b} chính là chứng minh phương trình Poisson $\Delta V = \frac{\rho}{\varepsilon_0}$ đối với hệ toạ độ trụ có tính đối xứng trụ, đây là một phương trình được dẫn ra từ 2 trong 4 phương trình Maxwell, 1 phương trình là định luật Gauss về thông lượng của điện trường và 1 phương trình về lưu số của điện trường (liên hệ giữa điện trường và điện thế). Tổng quát hơn, với mọi bài toán về tĩnh điện và các hệ từ trường dừng, ta sẽ đều cần sử dụng 1 phương trình về thông lượng và 1 phương trình về lưu số để giải quyết chúng. \\
Bài toán diot chân không và định luật Child-Langumuir được lấy từ Bài tập giải sẵn "Diode chân không" trang 59-60 quyển \textit{Điện từ học 2} của Jean Marie Brébec (bộ sách PFIEV), bản dịch của Lê Băng Xương, Nhà xuất bản giáo dục Việt Nam. Bài toán được tham khảo thêm từ bài 2.53 trang 109 trong quyển sách \textit{Introduction to Electrodynamics} của David Griffiths cũng như các bài báo về "Child-Langumuir law". \\
Như ta có thể thấy, bài toán diode chân không này phổ biến nhất là trường hợp 2 bản tụ phẳng rộng có diện tích $S$ đặt song song cách nhau một khoảng $d$ như trong bài 2.53 quyển \textit{Introduction to Electrodynamics}. Kết quả của bài toán này sẽ là:
$$ I = \frac{4 \varepsilon_0 S}{9d^2} \sqrt{ \frac{2e}{m}} U^{3/2}.$$
Các vấn để mở rộng cho bài toán này như khảo sát trường hợp electron có vận tốc ban đầu $v_0$ đáng kể cũng được được xét tới trong các bài báo \href{https://arxiv.org/abs/physics/0411175v1}{Generalization of Child-Langmuir Law for Non-Zero Injection Velocities in a Planar} và \href{http://de.arxiv.org/abs/1506.07417v1}{A new approach to the Child-Langmuir law}. 

Bài toán của chúng ta là dạng hình trụ của diode, tất nhiên, ta hoàn toàn có thể mở rộng bài toán này cho trường hợp hệ diode hình cầu, song nó đưa đến một phương trình vi phân tương đối phức tạp và không phù hợp để giải ở đây.