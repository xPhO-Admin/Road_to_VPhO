\textbf{1.} \textbf{a,} Sơ đồ tạo ảnh:
\begin{align}
     S_1 \xrightarrow[\displaystyle \infty \quad F]{\displaystyle F} S_2 \xrightarrow[\displaystyle L -F \quad l- f]{\displaystyle f_0} S_3 \xrightarrow[\displaystyle  f \quad \infty]{\displaystyle f} S_4.
\end{align}


Ta có phương trình tạo ảnh của thấu kính trường là
\begin{align}
\frac{1}{f_0} &= \frac{1}{L - F} + \frac{1}{d_0 - f}\\
\Leftrightarrow L &= F + \frac{f_0 (l - f)}{l - f - f_0} = 1177.5 \mathrm{~mm}.\label{41}
\end{align}

\textbf{b,} Độ bội giác của hệ ba thấu kính là
\begin{align}
    G_\infty = \frac{F}{f} \frac{l - f}{L - F} = \frac{F}{f} \left( 1 + \frac{f-l}{f_0} \right)  = \frac{160}{3} \approx 53.33.
\end{align}

\textbf{2.} Ta có sơ đồ tạo ảnh của \textit{vật kính} qua hệ thị kính:
\begin{align}
     S_1 \xrightarrow[\displaystyle L \quad d']{\displaystyle f_0} S_2 \xrightarrow[\displaystyle l - d' \quad \Delta]{\displaystyle  f} S_3.
\end{align}

Ta có hệ phương trình thấu kính:
\begin{align}
    \frac{1}{f_0} &= \frac{1}{L} + \frac{1}{d'}.\\
    \frac{1}{f} &= \frac{1}{l - d'} + \frac{1}{\Delta}.
\end{align}

Kết hợp với phương trình (\ref{41}) và giải hệ ba phương trình ta thu được kết quả
\begin{align}
    \Delta = f \left[\frac{1}{1 + \dfrac{f}{f_0 - l}} + \frac{f}{F} \frac{1}{\left(1 + \dfrac{f-l}{f_0} \right)^2} \right] =\frac{507}{64} \approx 7.92 \mathrm{~mm}.
\end{align}

Đường kính vòng tròn ánh sáng xuất hiện trên đồng tử mắt người là
\begin{align}
    \delta = -\frac{D}{G_\infty} = -\frac{39}{16} = -2.4375 \mathrm{mm}.
\end{align}

Ta thấy rằng $\delta$ bé hơn kích thước đồng tử mắt người, nên người có thể nhìn được toàn bộ ánh sáng qua hệ.

\vspace{1mm}

\textbf{3.} Để xác định thị trường qua kính thiên văn, ta coi mắt người là nguồn sáng điểm rồi tìm ảnh của mắt qua hệ thấu kính. Ta có sơ đồ tạo ảnh
\begin{align}
     S_1 \xrightarrow[\displaystyle \Delta \quad d_1]{\displaystyle f} S_2 \xrightarrow[\displaystyle l - d_1 \quad d_2]{\displaystyle f_0} S_3 \xrightarrow[\displaystyle  L - d_2 \quad d_3]{\displaystyle F} S_4.
\end{align}
Ta có hệ phương trình thấu kính
\begin{align}
    \frac{1}{f} &= \frac{1}{\Delta} + \frac{1}{d_1}.\\
    \frac{1}{f_0} &= \frac{1}{l - d_1} + \frac{1}{d_2}.\\
    \frac{1}{F} &= \frac{1}{L - d_2} + \frac{1}{d_3}.
\end{align}
Bấm máy thay số liệu ta thu được
\begin{align}
    d_1 &= -7.5\mathrm{~mm}\\
    d_2 &= -225 \mathrm{~mm}\\
    d_3 &=8311.11 \mathrm{~mm}
\end{align}

Giả sử từ mắt nhìn qua được toàn bộ thấu kính mắt. Khi đó đường kính vòng tròn nhìn được trên kính trường là $d_0'$, khi đó
\begin{align}
    \Big| \frac{d_1}{l -d_1} \Big| &= \frac{d}{d_0'}\\
    \Rightarrow d_0' &= 75 \mathrm{~mm} > d_0 = 30 \mathrm{~mm}
\end{align}

Nhận thấy rằng mắt người không thể nhận toàn bộ ánh sáng qua kính mắt.

Làm tương tự đối với kính trường, ta tìm được vòng tròn ánh sáng trên vật kính, gọi là $D'$, khi đó
\begin{align}
    \Big|\frac{d_2}{L - d_2} \Big| &= \frac{d_0}{D'}\\
    \Rightarrow D' &= 187 \mathrm{~mm} > D =130 \mathrm{mm}
\end{align}

Từ đó ta nhận định rằng ánh sáng đi qua toàn bộ vật kính sẽ không bị mất mát năng lượng khi đi đến mắt người, và \textit{không ngược lại}.

\vspace{1mm}
Từ đó ta tìm được góc mở thị trường khi nhìn qua kính thiên văn là 
\begin{align}
    \theta \approx \frac{D}{d_3} = 0.01564\mathrm{~rad}.
\end{align}
Đường kính góc của Mặt Trăng khi nhìn từ Trái Đất là 
\begin{align}
    \theta_M = \frac{2R_M}{d_{ME}} = 0.00900 \mathrm{~rad}.
\end{align}
Cuối cùng, ta xác định được tỉ lệ diện tích của Mặt Trăng so với trường nhìn qua kính thiên văn là
\begin{align}
    k = \left( \frac{\theta_M}{\theta} \right)^2 = 33.4 \%.
\end{align}



 \textbf{Biểu điểm} 
\begin{center}
\begin{tabular}{|>{\centering\arraybackslash}m{1cm}|>{\raggedright\arraybackslash}m{14cm}| >{\centering\arraybackslash}m{1cm}|}
    \hline
    \textbf{Phần} & \textbf{Nội dung} & \textbf{Điểm} \\
    \hline
    \textbf{a} & Viết được biểu thức chữ và giá trị số của $L$ & 0.50\\   
    \cline{2-3}
    &  Viết được biểu thức chữ và giá trị số của $G_\infty$& 0.50\\
    \hline
    \textbf{b} & Tìm được biểu thức chữ và giá trị của $\Delta$ & 1.00 \\
        \cline{2-3}
        & Tìm được $\delta$ & 0.50 \\
    \hline
    \textbf{c} & Tìm được $d_1,d_2,d_3$ & 0.50\\
    \cline{2-3}
    & Chứng minh được toàn bộ ánh sáng qua vật kính khi đến mắt không bị mất mát & 0.50\\
    \cline{2-3}
    & Tìm được tỉ số $k$ & 0.50\\
    \hline
\end{tabular}
\end{center}