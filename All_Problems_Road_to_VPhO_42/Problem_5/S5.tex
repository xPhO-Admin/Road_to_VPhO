\textbf{a,} Trong hệ quy chiếu khối tâm của hệ hạt, tổng động lượng của hai hạt bằng 0, hai hạt va vào nhau với động lượng bằng nhau, trực diện và sau va chạm tạo ra hạt $\Delta$ đứng yên (do bảo toàn động lượng).

Sử dụng định luật bảo toàn động lượng, ta có
\begin{align}
    p^\prime_\gamma c= p^\prime_p c = \sqrt{E^{\prime 2}_p - m_p^2 c^4}. 
\end{align}

Sử dụng định luật bảo toàn năng lượng, ta có 
\begin{align}
  m_\Delta c^2   &=  E^{\prime}_p + E^\prime_\gamma\\
   m_\Delta c^2   &=E^{\prime}_p +  \sqrt{E^{\prime 2}_p - m_p^2 c^4}\\
   (  m_\Delta c^2  - E^{\prime}_p )^2 &= E^{\prime 2}_p - m_p^2 c^4\\
   m_\Delta^2 c^4 - 2m_\Delta c^2 E^{\prime}_p & = - m_p^2 c^4\\
   \Rightarrow E^\prime_p &= \frac{m_\Delta^2 + m_p^2}{2 m_\Delta} c^2. \label{50}
\end{align}

\textbf{b,} Từ định luật bảo toàn động lượng ta có
\begin{align}
    p_\Delta^2 = p_p^2 + p_\gamma^2 + 2p_p p_\gamma \cos \theta.
\end{align}

Sử dụng định luật bảo toàn năng lượng, ta có
\begin{align}
    E_p + E_\gamma &= \sqrt{p_\Delta^2 c^2 + m_\Delta^2 c^4}\\
    E_p^2 + 2 E_p E_\gamma + E_\gamma^2 &= m_\Delta^2 c^4 + p^2_p c^2 + E_\gamma^2 + 2 E_\gamma p_p c \cos \theta\\
    2 E_\gamma (E_p - p_p c \cos \theta)&= (m_\Delta^2 - m_p^2) c^4. \label{51}
\end{align}

\textit{Cách khác:} Ta có thể sử dụng quy tắc \textit{vector 4 chiều} trong va chạm tương đối tính như sau
\begin{align}
    (E_\gamma, \vec{p_\gamma}) + (E_p, \vec{p}_p) &= (E_\Delta,\vec{p}_\Delta)\\
    (E_\gamma, \vec{p_\gamma})^2 + (E_p, \vec{p}_p)^2 + 2(E_\gamma, \vec{p_\gamma})(E_p, \vec{p}_p)&=(E_\Delta, \vec{p}_\Delta)^2\\
    0 + m_p^2 c^4 + 2(E_p E\gamma - \vec{p}_p \vec{p_\gamma} c^2 )&= m_\Delta^2 c^4.
\end{align}

Cũng trùng với phương trình (\ref{51}).

Với trường hợp $\theta = \pi$, tức va chạm trực diện, từ (\ref{51}) ta sẽ có 
\begin{align}
    E_p + \sqrt{E_p^2 - m_p^2 c^4} = \frac{m_\Delta^2 - m_p^2}{2E_\gamma} c^4
\end{align}

Chuyển phần tử căn thức thành vế phương trình độc lập rồi bình phương, biến đổi tiếp thu được
\begin{align}
    E_p (\pi) = \frac{(m_\Delta^2 - m_p^2) c^4}{4 E_\gamma}  + \frac{m_p^2 E_\gamma}{m_\Delta^2 - m_p^2}. \label{52}
\end{align}

Đối với trường hợp $\theta$ bất kì, làm tương tự ta sẽ ra được phương trình bậc hai đối với $E_p$ là
\begin{align}
    E_p^2 \sin^2 \theta -  \frac{(m_\Delta^2 - m_p^2) c^4}{E_\gamma} E_p + \left[  \left( \frac{(m_\Delta^2 - m_p^2) c^4}{2 E_\gamma} \right)^2 + m_p^2 c^4 \cos^2 \theta \right] = 0. \label{53}
\end{align}


Giải phương trình ta thu được nghiệm thoả mãn
\begin{align}
    E_p (\theta) = \frac{(m_\Delta^2 - m_p^2) c^4}{2 E_\gamma \sin^2 \theta} \left(1+ \sqrt{ \cos^2 \theta - \frac{m_p^2 E_\gamma^2 \sin^2 2\theta}{(m_\Delta^2 - m_p^2)^2 c^4} } \right).
\end{align}

Lưu ý rằng phương trình trên chỉ có nghiệm khi phần tử trong căn thức là dương. Tức là
\begin{align}
    \cos^2 \theta - \frac{m_p^2 E_\gamma^2 \sin^2 2\theta}{(m_\Delta^2 - m_p^2)^2 c^4}& > 0
\end{align}
Thu được điều kiện với góc mở của hướng hai hạt
\begin{align}
    \sin \theta < \frac{(m_\Delta^2 - m_p^2)c^2}{2 m_p E_\gamma}.
\end{align}


 \textbf{Biểu điểm} 
\begin{center}
\begin{tabular}{|>{\centering\arraybackslash}m{1cm}|>{\raggedright\arraybackslash}m{14cm}| >{\centering\arraybackslash}m{1cm}|}
    \hline
    \textbf{Phần} & \textbf{Nội dung} & \textbf{Điểm} \\
    \hline
    & Nhận xét ra $\vec{p} =0$ trong HQC khối tâm & 0.5\\   
     \cline{2-3}
    \textbf{a} &  Viết được hai phương trình bảo toàn $E, \vec{p}$ & 0.5\\
    \cline{2-3}
    & Biểu diễn đúng $E^\prime_p$ (\ref{50}) & 0.5\\

    \hline
    
     & Dẫn ra được phương trình (\ref{51}) & 1\\
     \cline{2-3}
    \textbf{b} & Biểu diễn đúng $E_p (\pi)$ (\ref{52}) & 1\\
    \cline{2-3}
    & Biểu diễn đúng $E_p (\theta)$ (\ref{53}) & 0.5\\
    \hline
\end{tabular}
\end{center}