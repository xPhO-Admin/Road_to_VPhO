\textbf{Bức xạ vũ trụ} hay \textbf{tia vũ trụ} (viết tắt là CR-\textit{Cosmic ray}) là chùm tia các hạt photon hoặc hạt nhân nguyên tử có năng lượng cao phóng vào khí quyển Trái Đất từ không gian (bức xạ sơ cấp) và bức xạ thứ cấp được sinh ra do các hạt đó tương tác với các hạt nhân nguyên tử trong khí quyển với thành phần gồm hầu hết là các hạt cơ bản. Bức xạ vũ trụ sơ cấp đẳng hướng trong không gian và không đổi theo thời gian. Bức xạ vũ trụ có tính sát thương mạnh. Theo thiên văn học hiện đại, vũ trụ chứa đầy bức xạ điện từ còn sót lại sau vụ nổ Big Bang gọi là bức xạ nền vũ trụ hay bức xạ phông vi sóng vũ trụ (viết tắt là CMB-\textit{Cosmic microwave background}). Xét sự tương tác giữa CR và CMB, được đơn giản thành phản ứng 
$$p + \gamma \rightarrow \Delta.$$

Với $p$ là hạt proton của chùm CR, $\gamma$ là photon tàn dư của CMB, $\Delta$ là baryon nhẹ nhất (khi so sánh với các nucleon khác) có khối lượng nghỉ $m_\Delta = 1232  \mathrm{~MeV/c^2}$. Biết rằng hướng chuyển động của hai chùm tia tạo với nhau một góc $\theta$.

\begin{enumerate}[label=\textbf{\alph*,}]\itemsep0em
    \item Hãy tìm năng lượng $E_p^\prime$ của proton trong hệ quy chiếu khối tâm của hệ hạt.
    \item Hãy tìm năng lượng $E_p$ của proton trong hệ quy chiếu Thiên Hà theo các đại lượng $E_\gamma$, $m_p$, $m_\Delta$, $\theta$ và $c$. Tính trong trường hợp $\theta$ bất kì và $\theta = \pi$ (va chạm trực diện).
\end{enumerate}


\textit{Gợi ý: Trong mọi hệ quy chiếu quán tính, đại lượng $E^2 - |\Vec{p}|^2 c^2 = m_0^2 c^4$ là một đại lượng bất biến. Trong đó $E$, $\Vec{p}$ và $m_0$ lần lượt là năng lượng, động lượng và khối lượng nghỉ của một hạt chuyển động tương đối tính.}

\begin{flushright}
    (Biên soạn bởi Zinc)
\end{flushright}