Cho $\SI{1}{mol}$ khí lý tưởng có hằng số đoạn nhiệt $\gamma$ thực hiện một chu trình thuận nghịch gồm 8 đường đẳng nhiệt và 8 đường đoạn nhiệt xen kẽ (đường bắt đầu chu trình là đường đẳng nhiệt), trong đó, 7 đường giãn đoạn nhiệt và 7 đường giãn đẳng nhiệt, 1 đường nén đẳng nhiệt và 1 đường nén đoạn nhiệt đưa khối khí về trạng thái ban đầu. Trong 7 quá trình trong giãn đẳng nhiệt, thế tích khối khí tăng lần lượt $n_1=1.10$, $n_2=1.09$, $n_3=1.08$, $n_4=1.07$, $n_5=1.06$, $n_6=1.05$, $n_7=1.04$ lần. Nhiệt độ của 8 quá trình đẳng nhiệt lần lượt là $T_1=\SI{373}{K}$, $T_2=\SI{363}{K}$, $T_3=\SI{353}{K}$, $T_4=\SI{343}{K}$, $T_5=\SI{333}{K}$, $T_6=\SI{323}{K}$, $T_7=\SI{313}{K}$, $T_8=\SI{303}{K}$. Cho hằng số khi lý tưởng là $R=\SI{8.31}{J/K \cdot mol}$.

\begin{enumerate}[label=\textbf{\alph*,}]\itemsep0em
\item Entropy $S$ là một hàm trạng thái trong nhiệt động lực học. Với các quá trình biến đổi thuận nghịch, khi người ta cấp cho một vật nhiệt độ $T$ một nhiệt lượng $\delta Q$, ta có biểu thức
$$ dS = \frac{\delta Q}{T}.$$
Hãy chỉ ra rằng entropy $S$ của $\SI{1}{mol}$ khí lý tưởng là một hàm trạng thái mà chỉ phụ thuộc vào nhiệt độ $T$ và thể tích $V$ của khối khí. 

\item Vẽ phác đồ thị chu trình của bài toán trên giản đồ $T$-$S$ (giản đồ có trục tung là nhiệt độ $T$ và trục hoành là entropy $S$ của khối khí).

\textit{Ở phần này, chỉ yêu cầu vẽ phác hình dạng của đồ thị, không yêu cầu vẽ chính xác số liệu hay tỷ lệ.}

\item Trong quá trình nén đẳng nhiệt duy nhất của chu trình, thể tích khối khí đã giảm đi bao nhiêu lần?

\item Tính công khí thực hiện trong một chu trình và hiệu suất của chu trình.
\end{enumerate}

\begin{flushright}
    (Biên soạn bởi Log)
\end{flushright}