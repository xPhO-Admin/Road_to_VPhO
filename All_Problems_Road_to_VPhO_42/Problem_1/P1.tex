Một quả lựu đạn đang đứng yên ở độ cao $H$ so với mặt đất thì phát nổ và bắn ra các mảnh đạn giống nhau với cùng tốc độ $v_0$, phân bố các mảnh đạn khi văng ra là đẳng hướng. Gia tốc trọng trường là $\Vec{g}$.
\begin{enumerate}[label=\textbf{\alph*,}]\itemsep0em
    \item Xác định phương trình mặt cong ba chiều biểu diễn ranh giới giữa \textit{vùng an toàn} và \textit{vùng nguy hiểm}, tức là trong vùng an toàn thì mảnh đạn sẽ không thể bay tới, ngược lại với vùng nguy hiểm. Vẽ phác dạng đồ thị của đường ranh giới, ghi chú thích và các điểm đặc biệt.
    \item Sau khi các mảnh đạn rơi hết xuống đất, xác định bán kính $R$ của vùng đạn đã rơi trên mặt đất.
    \item Giả sử rằng lựu đạn chứa $M$ khối lượng đạn nổ trong một góc nón nhỏ hướng lên trên với góc mở $2\alpha_0 \ll 1$, bom nổ trên mặt đất ($H = 0$). Chứng minh rằng phân bố khối lượng mặt $\rho(r)$ (khai triển tới thành phần bậc hai $r^2$) của đạn trên mặt đất khi tất cả mảnh đạn đã rơi xuống đất có dạng như sau
    $$\rho(r) \approx \rho_0 (1 + \beta r^2),$$

    hãy biểu diễn $\rho_0$ và $\beta$ theo các thông số đã biết.
\end{enumerate}

\textit{Có thể sử dụng xấp xỉ sau:} $\displaystyle \sin \alpha \approx \alpha \left(1 - \frac{1}{6} \alpha^2 \right)$.

\begin{flushright}
    (Biên soạn bởi Zinc)
\end{flushright}