\textbf{a,}
Khi hình trụ ổn định nhiệt, công suất tỏa nhiệt Joule khi dòng điện chạy qua bằng công suất tỏa nhiệt ra môi trường.


Công suất tỏa nhiệt Joule trên toàn bộ dây dẫn là
\begin{equation} \label{eq1_p2_d2}
\begin{split}
P_j = I^2 R &= I^2 \frac{\rho l}{\pi r^2}\\
&= I^2 \frac{\rho_0 l_0}{\pi r_0^2} [1 + (\beta - \alpha) (T-T_0)].
\end{split}
\end{equation}
Ở đây ta sử dụng liên tiếp các bước xấp xỉ tuyến tính (v.d $(1 + \beta)(1 + \alpha) \approx 1 + \beta+\alpha$ với $\beta$ và $\alpha$ nhỏ).


Công suất tỏa nhiệt ra môi trường trên toàn dây dẫn là
\begin{equation} \label{eq2_p2_d2}
\begin{split}
P_l &= \lambda S (T-T_0)\\
&= \lambda \cdot 2\pi rl (T-T_0)\\
&= 2\pi\lambda r_0 l_0 (T-T_0) [1 + 2\alpha (T - T_0)].
\end{split}
\end{equation}

Điều kiện cân bằng nhiệt
\begin{equation} \label{eq3_p2_d2}
I^2 \frac{\rho_0 l_0}{\pi r_0^2}\frac{1 + (\beta - \alpha)(T-T_0)}{1 + 2\alpha (T-T_0)} = 2\pi\lambda r_0 l_0 (T-T_0).
\end{equation}

Bỏ qua thành phần bậc 2 của $\beta$ và $\alpha$ có được
\begin{align} \label{eq4_p2_d2}
T_f = T_0 + \frac{I^2 \rho_0}{2\pi^2 \lambda r_0^3} \left(1 + \frac{I^2 \rho_0}{2\pi^2 \lambda r_0^3} (\beta - 3 \alpha) \right)
\end{align}

\textbf{b, }
Ký hiệu dòng nhiệt là $\vec{j} = j \hat{x}$.

Phương trình truyền nhiệt theo định luật Fourier
\begin{align}
j = - k \frac{\partial T}{\partial x}.
\end{align}

Ở đây ta xét một lát cắt của hình trụ có chiều dài $dx$ tại tọa độ $x$.
Thông lượng nhiệt đi ra khỏi lát cắt này là
\begin{equation} \label{eq5_p2_d2}
\begin{split}
d P_{loss} &= \lambda \cdot 2\pi r_0 d x (T - T_0) + j(x + dx) \pi r_0^2 - j(x) \pi r_0^2\\
&= \left[2\pi\lambda r_0 (T-T_0) + \pi r_0^2 \frac{\partial j}{\partial x}\right] dx.
\end{split}
\end{equation}

Nhiệt năng Joule tỏa ra tại lát cắt này là
\begin{equation} \label{eq6_p2_d2}
d P_j = I^2 \frac{\rho d x}{\pi r_0^2}.
\end{equation}

Tại chế độ dừng, hai lượng nhiệt này phải bằng nhau. Khi đó
\begin{equation} \label{eq7_p2_d2}
\begin{split}
2\pi\lambda r_0 (T-T_0) + \pi r_0^2 \frac{\partial ^2 (T-T_0)}{\partial x^2} = I^2 \frac{\rho}{\pi r_0^2}\\
\Rightarrow \frac{\partial ^2}{\partial x^2} (T-T_0) - \frac{2\lambda}{kr_0} (T-T_0) + \frac{I^2 \rho_0}{k\pi^2 r_0^4} = 0.
\end{split}
\end{equation}

Phương trình này có nghiệm tổng quát là
\begin{equation} \label{eq8_p2_d2}
T(x)= T_0 + \frac{I^2 \rho_0}{2\lambda\pi^2 r_0^3}\left(1 + C_1 e^{\mu (x-x_0)} + C_2 e^{-\mu (x-x_0)}\right).
\end{equation}
Với $\lambda = \sqrt{\frac{2\lambda}{kr_0}}$, $C_1, C_2$ là các đại lượng không thứ nguyên, $x_0$ là một hằng số phụ thuộc vào điều kiện đầu.

Giải hệ phương trình điều kiện biên $ T(0)=T(l)=T_{0} $, ta sẽ thu được giá trị của $ C_{1},C_{2} $. Từ đó ta có nghiệm
\begin{equation} \label{eq9_p2_d2}
T(x)= T_0 + \frac{I^2 \rho_0}{2\lambda\pi^2 r_0^3}\left(1 - \frac{\cosh[\mu (x-x_0)]}{\cosh(\mu l_0/2)}\right),
\end{equation}
với $\cosh$ là hàm cos hyperbolic, $\displaystyle \cosh x = \frac{e^{x} + e^{-x}}{2}$. 


\textbf{Biểu điểm} 
\begin{center}
\begin{tabular}{|>{\centering\arraybackslash}m{1cm}|>{\raggedright\arraybackslash}m{14cm}| >{\centering\arraybackslash}m{1cm}|}
    \hline
    \textbf{Phần} & \textbf{Nội dung} & \textbf{Điểm} \\
    \hline
    \textbf{a} & Tìm được công suất tỏa nhiệt Joule (\ref{eq1_p2_d2})  & 0.50\\   
    \cline{2-3}
    &  Tìm được công suất tỏa nhiệt (\ref{eq2_p2_d2}) & 0.50 \\
    \cline{2-3}
    & Khai triển tìm được $T - T_0$ (\ref{eq4_p2_d2}) & 0.50\\
    \hline
    \textbf{b} & Tìm được thông lượng nhiệt tại vi phân thể tích (\ref{eq5_p2_d2}) & 0.50\\
    \cline{2-3}
    & Tìm được nhiệt lượng tỏa ra tại vi phân thể tích (\ref{eq6_p2_d2}) & 0.25\\
    \cline{2-3}
    & Đưa về dạng phương trình vi phân và đưa ra dạng nghiệm tổng quát (\ref{eq7_p2_d2}) & 1.00 \\
    \cline{2-3}
    & Tìm được nghiệm riêng phù hợp điều kiện biên (\ref{eq9_p2_d2}) & 0.75 \\
    \hline
\end{tabular}
\end{center}

