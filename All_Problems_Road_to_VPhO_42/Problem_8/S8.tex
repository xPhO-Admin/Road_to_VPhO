\textbf{a,} Chọn hệ tọa độ trụ ($r$, $\phi$, $z$) sao cho chiều tăng của $\phi$ trùng với chiều dòng điện qua trụ.

Ta có từ trường bên trong trụ là đều và bằng:
\begin{equation} \label{eq1_p3_d2}
\vec{B} = \mu_0 \frac{i}{H} \hat{z},
\end{equation}
với $i$ là dòng điện tức thời trên trụ.

Do đối xứng, khi từ trường tăng lên khi dòng điện tăng, sẽ có điện trường xoáy theo phương $\hat{\boldsymbol{\phi}}$. 
Điện trường này sẽ gây một xung lượng lên các hạt điện tích trong khoảng không bên trong trụ.

Theo định luật Faraday
\begin{equation} \label{eq2_p3_d2}
\begin{split}
2\pi r E(r) = \pi r^2 \frac{\partial B}{\partial t}\\
\Rightarrow E = \frac{\mu_0 r}{2 H} \frac{\partial i}{\partial t}.
\end{split}
\end{equation}

Theo định luật Lenz, $E$ hướng ngược chiều $\hat{\boldsymbol{\phi}}$ (là chiều dòng điện gây ra biến thiên từ trường) (giả sử $\displaystyle \frac{\partial i}{\partial t} > 0$).

Có tốc độ mà một hạt tích điện nhận được
\begin{equation} \label{eq3_p3_d2}
\begin{split}
\vec{v}_0 &= \lim_{\Delta t \to 0} \int_0^{\Delta t} \frac{\vec{F}}{m} dt \\
&= - \lim_{\Delta t \to 0} \int_{0}^{I} \frac{\mu_0 qr}{2mH}\hat{\boldsymbol{\phi}} di\\
&= -\frac{\mu_0 qrI}{2mH}\hat{\boldsymbol{\phi}}.
\end{split}
\end{equation}

\textbf{b,} Khi dòng điện đã ổn định, các điện tích chỉ chuyển động trong từ trường.
Do đó tốc độ từng hạt được bảo toàn.

Do từ trường hướng theo trục $\hat{z}$, vận tốc đầu vuông góc với trục $\hat{z}$ nên chuyển động sau đó của hạt nằm trong mặt phẳng vuông góc với $\hat{z}$.

Ký hiệu $R_0$ là tọa độ $r$ ban đầu của hạt ta đang xét.

Như vậy
\begin{equation} \label{eq4_p3_d2}
\begin{split}
\vec{v} &= \dot{r} \hat{r} + r\dot{\phi} \hat{\boldsymbol{\phi}}\\
\Rightarrow |\vec{v}| &= \sqrt{\dot{r}^2 + r^2 \dot{\phi}^2} = v_0.
\end{split}
\end{equation}

Ta nhận thấy trong trường hợp này, momen động lượng của một hạt tích điện (so với trục $z$) không được bảo toàn.
Theo đó
\begin{equation} \label{eq5_p3_d2}
\begin{split}
\frac{d \vec{L}}{d t} &= \vec{r} \times (q \vec{v} \times \vec{B})= q[(\vec{r} \cdot \vec{B})\vec{v} - (\vec{r} \cdot \vec{v})\vec{B}]= -\frac{q\vec{B}}{2} \frac{\partial (r^2)}{\partial t}.
\end{split}
\end{equation}

\begin{equation} \label{eq6_p3_d2}
\begin{split}
\Rightarrow \vec{L} + \frac{qr^2 \vec{B}}{2} &= \mathrm{const}\\
&= -mR_0 v_0 \hat{z} + \frac{qR_0^2 B}{2} \hat{z}\\
&= 0.
\end{split}
\end{equation}
Từ đó tìm được vận tốc $v_{\phi}$ theo phương tiếp tuyến (không xét chiều)
\begin{equation} \label{eq7_p3_d2}
\begin{split}
    v_{\phi} &= \frac{qr^2 B}{2mr}\\
&= \frac{qB}{2m}r.
\end{split}
\end{equation}
Do đó vận tốc theo phương tiếp tuyến $\dot{r}$
\begin{equation} \label{eq8_p3_d2}
\begin{split}
|\dot{r}| &= \sqrt{v_0^2 - \left(\frac{qB}{2m}\right)^2 r^2} \\
&= \frac{qB}{2m} \sqrt{R_0^2 - r^2}.
\end{split}
\end{equation}
Dễ dàng nhận thấy để biểu thức có nghĩa, $r < R_0$ trong suốt quá trình chuyển động. Do đó $\dot{r}$ mang dấu âm.\\
Khoảng thời gian từ khi dòng điện được bật lên cho đến khi vật đang xét đập vào thành trong của ống trụ là
\begin{equation} \label{eq9_p3_d2}
\begin{split}
\tau &= \int_{R_0}^{R_1} -\frac{2m}{qB} \frac{d r}{\sqrt{R_0^2 - r^2}}\\
&= \frac{2m}{qB} \arccos \left(\frac{R_1}{R_0}\right).
\end{split}
\end{equation}

\textbf{c,} Chú ý rằng với $t$ nhỏ, tác động do các hạt có khối lượng bé hơn sẽ là chủ đạo (khối lượng electron nhỏ hơn proton khoảng 2000 lần).\\
Tại thời điểm $t$, các hạt tới từ lớp $r = R(t)$ sẽ đập vào thành trong trụ với $R(t)$ bằng
\begin{equation} \label{eq10_p3_d2}
R(t) = \frac{R_1}{\cos\left(\dfrac{qBt}{2m}\right)}.
\end{equation}
Động lượng mà một hạt truyền cho thành trong trụ khi đập vào là
\begin{equation} \label{eq11_p3_d2}
\begin{split}
p_1 &= m \dot{r}= \frac{qB}{2} \sqrt{R_1^2 \left(\frac{1}{\cos^2 \left(\frac{qBt}{2m}\right)}-1\right)}\\
&= \frac{qBR_1}{2} \tan\left( \dfrac{qBt}{2m}\right).
\end{split}
\end{equation}

Áp suất tác dụng lên thành trong của trụ
\begin{equation} \label{eq12_p3_d2}
\begin{split}
p &= \frac{1}{2\pi R_1 l} \frac{qBR_1}{2} \tan \left(\frac{qBt}{2m}\right) n \cdot 2\pi R(t) \frac{d R}{d t},
\end{split}
\end{equation}
với $R(t)$ là hàm được định nghĩa ở trên.
Đặt $\displaystyle \omega = \frac{qB}{2m}$. 

Ta tìm được áp suất là
\begin{equation} \label{eq13_p3_d2}
p = nm \omega^2 R_1^2 \frac{\sin^2(\omega t)}{\cos^4(\omega t)}.
\end{equation}

\textbf{d,} Ta thấy $p$ là hàm đồng biến theo $t$ (khi $\displaystyle \omega t < \frac{\pi}{2}$).
\begin{equation} \label{eq14_p3_d2}
\begin{split}
p_{max} &= nm \omega ^2 R_2^2 \left[\left(\frac{R_2}{R_1}\right)^2 - 1\right]\\
\Rightarrow \beta &= \frac{p_{max}}{\omega_B} = \frac{\mu_0 n e^2}{2m} R_2^2 \left[\left(\frac{R_2}{R_1}\right)^2 - 1\right].   
\end{split}
\end{equation}
Thế số, ta được $\beta = 1.77 > 1$.

\textbf{Biểu điểm} 
\begin{center}
\begin{tabular}{|>{\centering\arraybackslash}m{1cm}|>{\raggedright\arraybackslash}m{14cm}| >{\centering\arraybackslash}m{1cm}|}
    \hline
    \textbf{Phần} & \textbf{Nội dung} & \textbf{Điểm} \\
    \hline
    \textbf{a} & Tìm được từ trường bên trong trụ (\ref{eq1_p3_d2})  & 0.10\\   
    \cline{2-3}
    &  Tìm được điện trường bên trong trụ theo định luật Faraday (\ref{eq2_p3_d2}) & 0.10 \\
    \cline{2-3}
    & Chú ý rằng vẫn truyền một xung lượng cho hạt tích điện khi $\Delta t \to 0$ & 0.10\\
    \cline{2-3}
    & Tìm được vận tốc truyền cho hạt tích điện (\ref{eq3_p3_d2}) & 0.20\\
    \hline
    \textbf{b} & Nhận ra rằng tốc độ hạt là không đổi và tìm được thành phần vận tốc theo phương bán kính & 0.25\\
    \cline{2-3}
    & Đạo hàm momen động lượng và đưa ra phương trình (\ref{eq6_p3_d2}) (dạng vector hoặc vô hướng) & 0.75\\
    \cline{2-3}
    & Đưa về dạng phương trình vi phân tại (\ref{eq8_p3_d2}) & 0.25 \\
    \cline{2-3}
    & Tích phân để tìm được khoảng thời gian hạt đập vào thành trong (\ref{eq9_p3_d2}) & 0.25 \\
    \hline
    \textbf{c} & Tìm được bán kính đầu của các hạt đập vào thành trong trụ tại thời điểm $t$ (\ref{eq10_p3_d2}) & 0.25\\
    \cline{2-3}
    & Tìm được động lượng mà hạt truyền cho trụ (\ref{eq11_p3_d2}) & 0.50\\
    \cline{2-3}
    & Tìm được dạng cơ bản của áp suất tác dụng lên thành trong trụ (\ref{eq12_p3_d2}) & 0.50 \\
    \cline{2-3}
    & Khai triển và đưa về dạng hoàn chính (\ref{eq13_p3_d2}) & 0.25\\
    \hline
    \textbf{d} & Tìm được $P_{max}$ theo $R_2 / R_1$ (\ref{eq14_p3_d2}) & 0.25\\
    \cline{2-3}
    & Tìm được giá trị (bằng số và biểu thức) của $\beta$ & 0.25\\
    \hline
\end{tabular}
\end{center}