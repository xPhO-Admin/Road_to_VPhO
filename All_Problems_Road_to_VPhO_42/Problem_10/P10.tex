\textbf{1.} Ở vật lý sơ cấp, chúng ta đã biết đến nhiều cách đo \textit{điện tích riêng} của một hạt (phổ biến nhất là electron) bằng cách sử dụng điện từ trường. Tuy nhiên, để đo chính xác điện tích (hoặc tương đương là khối lượng) thì không phổ biến đến thế. Ở đây chúng ta sẽ tìm hiểu phương pháp đo của nhà vật lý người Mỹ Robert Millikan (1886-1953).

\begin{enumerate}[label=\textbf{\alph*,}]\itemsep0em
\item 
Cho các dụng cụ như sau
\begin{itemize}\itemsep0em
\item Một bình kim loại có kích thước lớn để chứa các bản kim loại và kín để cô lập với không khí bên ngoài
\item Một đầu phun nhỏ giọt các giọt dầu được tích điện (điện tích, kích thước của từng giọt chưa được biết trước). Lưu ý đầu phun này còn đưa vào. bình các điện tích khác dưới dạng các hạt bụi có kích thước rất bé so với các giọt dầu.
\item Hai bản kim loại phẳng, có khoảng cách giữa chúng không đáng kể so với kích thước. Cho một số lỗ trên hai bản này và giả thiết các lỗ này không ảnh hướng đến điện trường giữa hai bản.
\item Các máy đếm thời gian gắn với cảm biến quang (thay thế cho các ống nhìn và đồng hồ bấm tay).
\item Một nguồn pin DC 5000V, với độ sụt áp không đáng kể so với thời gian thực hiện thí nghiệm.
\item Các dây nối, thước đo phù hợp.
\end{itemize}

Cho biết gia tốc trọng trường $g$, khối lượng riêng của dầu $\rho$, hằng số khí lý tưởng $R$, quãng đường tự do trung bình của không khí tại nhiệt độ bên trong bình $l$. Chú ý rằng các giọt dầu không giữ nguyên điện tích trong suốt quá trình đo do các hạt bụi tích điện trong không khí.  Công thức về lực cản của chất lưu Newton (ở đây là không khí loãng) đối với vật có dạng cầu lý tưởng:
\begin{align}
F = 6\pi \nu rv.
\end{align}

Hãy nêu phương án sử dụng để tính điện tích của electron. Đánh giá các tác nhân có thể gây ra sai số và nêu cách khắc phục (nếu có).

\item
Do kích thước rất bé của các giọt dầu (tỉ số $l/r$ có thể đạt 0.2-0.5), Millikan cần căn chỉnh lại định luật Stokes về lực cản của môi trường:
\begin{align}
F = 6\pi \nu rv \left(1 + A\frac{l}{r}\right),
\end{align}
với $A$ là một hằng số chưa xác định, $l$ là chiều dài tự do trung bình của môi trường (đã được cho trước).
Lưu ý rằng kết quả này chỉ là phân tích chuỗi Taylor đối với công thức Stokes đến bậc nhất của $l/r$ mà không theo kết quả lý thuyết nào.

Hãy điều chỉnh lại cách tính điện tích của electron cho phù hợp.

\item Tại thí nghiệm đầu tiên của mình, Millikan thu được bảng số liệu về các điện tích như sau. Hãy tìm điện tích của electron (đơn vị: esu, \textit{electrostatic unit} là một đơn vị đo điện tích). Không yêu cầu tìm sai số.
\vspace{-3mm}
\begin{center}
\begin{tabular}{|>{\centering\arraybackslash}m{1cm}|>{\centering\arraybackslash}m{6cm}|>{\centering\arraybackslash}m{1cm}|>{\centering\arraybackslash}m{6cm}|}
\hline
STT & Điện tích (esu) & STT & Điện tích (esu) \\
\hline
\hline
1 & 34.47 & 11 & 44.40\\
2 & 39.50 & 12 & 59.06\\
3 & 44.42 & 13 & 53.95\\
4 & 49.41 & 14 & 68.65\\
5 & 39.45 & 15 & 83.22\\
6 & 59.12 & 16 & 78.34\\
7 & 44.36 & 17 & 68.67\\
8 & 49.47 & 18 & 63.68\\
9 & 53.90 & 19 & 59.20\\
10 & 49.37 & 20 & 63.69\\
\hline
\end{tabular}
\end{center}
\end{enumerate}

\begin{flushright}
    (Biên soạn bởi Yukon)
\end{flushright}