%%IPhO Syllabus
%%Dịch bởi Yukon

\documentclass{article}
\usepackage[utf8]{inputenc}
\usepackage[vietnamese]{babel}
\usepackage{multicol}

\usepackage{titlesec}

\titleformat*{\section}{\Large\bfseries}
\titleformat*{\subsection}{\large\bfseries}
\titleformat*{\subsubsection}{\bfseries}

\usepackage[a4paper]{geometry}
\geometry{left = 1cm, right=1cm, top=1cm, bottom=1cm, noheadfoot, nomarginpar}
%\setlength{\topskip}{2cm}
%\setlength{\parindent}{2cm}

\pagestyle{empty}
\begin{document}

\begin{center}
    {\Huge\textbf{Danh mục kiến thức IPhO}}
\end{center}
\begin{multicols}{2}
\section{Giới thiệu}
\subsection{Mục đích của Danh mục}
Danh mục này thống kê các chủ đề được sử dụng trong IPhO. Độ khó của từng chủ đề trong danh sách có thể tham khảo từ đề IPhO các năm trước.

\subsection{Tính chất của các câu hỏi}
Các câu hỏi nên tập trung vào đánh giá sự sáng tạo và sự thông hiểu vật lý thay vì thử thách khả năng toán hoặc tốc độ làm việc. Phần điểm dành cho việc tính toán nên được giữ không quá lớn. Trong các ý yêu cầu tính toán khó, những lời giải xấp xỉ khác lời giải của đề nên nhận một số điểm tương đối. Đề nên được trình bày súc tích; giấy kiểm tra lý thuyết và thực hành nên ít hơn 1200 ký tự (bao gồm khoảng trắng, không tính bìa và phần trình bày bài).

\subsection{Ngoại lệ}
Các câu hỏi có thể bao gồm khái niệm và hiện tượng không được nhắc đến trong Danh mục này phải có đủ thông tin trong đề để thí sinh không có kiến thức từ trước không bị bất lợi vì điều này. Những khái niệm mới đó phải có liên hệ gần gũi với những chủ đề được nhắc đến trong Danh mục này. Những khái niệm này phải được giải thích theo các thuật ngữ trong Danh mục.

\subsection{Thứ nguyên}
Các giá trị bằng số phải sử dụng đơn vị SI, hoặc các hệ đơn vị được chấp nhận sử dụng chính thức cùng hệ SI.\\
Thí sinh được cho là có hiểu biết về các hiện tượng, khái niệm và phương thức được liệt kê dưới đây, và có thể vận dụng một cách sáng tạo.

\section{Kỹ năng lý thuyết}

\subsection{Tổng quan}
Kỹ năng tính xấp xỉ hợp lý khi mô hình hóa các hiện tượng đời thực. Khả năng nhận biết và khai thác sự đối xứng trong các câu hỏi.

\subsection{Cơ học}
\subsubsection{Động học}
(Biết được) vận tốc và gia tốc của chất điểm là các đạo hàm của vector độ dời. Tốc độ tiếp tuyến; gia tốc hướng tâm và tiếp tuyến. Chuyển động của một chất điểm với gia tốc không đổi. Phép cộng vận tốc và vận tốc góc; phép cộng gia tốc không tính đến thành phần Coriolis; nhận biết các trường hợp khi gia tốc Coriolis bằng 0. Chuyển động của một vật rắn như là chuyển động quay quanh tâm quay tức thời; vận tốc và gia tốc của các điểm trên vật rắn xoay.

\subsubsection{Tĩnh học}
Tìm khối tâm của một hệ bằng phép cộng hoặc tích phân. Điều kiện cân bằng: cân bằng lực (vector hoặc hình chiếu) và cân bằng moment (cho hình học 1 hoặc 2 chiều). Lực pháp tuyến, lực căng, lực ma sát nghỉ và ma sát trượt; định luật Hooke, áp lực, biến dạng và module Young. Cân bằng bền và không bền.

\subsubsection{Động lực học}
Định luật 2 Newton (ở dạng vector và hình chiếu); cơ năng chuyển động tịnh tiến và chuyển động quay. Thế năng cho các trường lực cơ bản (cũng như tích phân đường của trường lực). Moment, momen động lượng, năng lượng và các định luật bảo toàn. Công và công suất cơ học; tiêu tán năng lượng do ma sát. Hệ quy chiếu quán tính và phi quán tính: lực quán tính, lực hướng tâm, thế năng trong hệ quy chiếu quay. Moment quán tính của các vật đơn giản (vòng, đĩa, khối cầu, khối cầu rỗng, thanh), định lý trục song song, tìm moment quán tính bằng tích phân.

\subsubsection{Cơ thiên thể}
Định luật hấp dẫn, thế năng hấp dẫn, định luật Kepler (không cần chứng minh định luật I và III). Năng lượng của chất điểm trên quỹ đạo elipse.

\subsubsection{Cơ chất lưu}
Áp suất, lực đấy Archimedes, định luật liên tục, phương trình Bernoulli. Sức căng bề mặt và áp suất, năng lượng tương ứng.

\subsection{Điện từ trường}
\subsubsection{Khái niệm cơ bản}
Khái niệm về điện tích và dòng điện; bảo toàn điện tích và định luật dòng điện Kirchhoff. Lực Coulomb; trường tĩnh điện là trường thế; định luật điện thế Kirchhoff. Từ trường $\vec{B}$; lực Lorentz; lực Ampère; định luật Biot-Savart và trường $\vec{B}$ trên trục của một dòng điện kín hình tròn và hệ đối xứng đơn giản như dây thẳng, vòng dây hình tròn và solenoid dài.

\subsubsection{Dạng tích phân của các phương trình Maxwell}
Định luật Gauss (cho trường $\vec{E}$ và trường $\vec{B}$); định luật Ampère; định luật Faraday; sử dụng các định luật này để tìm trường khi hàm tích phân gần như là hằng số trên vi phân không gian. Điều kiện biên cho điện trường (hoặc điện thế) tại gần vật dẫn và tại vô cùng; khái niệm về vận dẫn nối đất. Nguyên lý chồng chất cho điện trường và từ trường; sự duy nhất của lời giải (nếu có); phương pháp ảnh điện.

\subsubsection{Tương tác của vật chất với điện trường và từ trường}
Điện dẫn suất và điện trở suất; dạng vi phân của định luật Ohm. Điện môi và độ từ thẩm; độ điện thẩm và từ thẩm của vật liệu điện và từ; mật độ năng lượng của điện trường và từ trường; vật liệu sắt từ; từ trễ và tiêu tán trong vật luật; dòng Eddy; định luật Lenz. Điện tích trong từ trường: chuyển động xoắn ốc, tần số cyclotron, vận tốc dạt trong trường $\vec{E}$ và $\vec{B}$ vuông góc. Năng lượng của lưỡng cực từ trong từ trường; moment lưỡng cực của dòng điện kín.

\subsubsection{Mạch điện}
Điện trở tuyến tính và định luật Ohm; định luật Joule; công của suất điện động, nguồn dòng, Ampere kế, Volt kế và Ohm kế. Linh kiện phi tuyến với đặc trưng $V-I$ cho trước. Tụ điện và điện dung (bao gồm của một điện cực so với vô cùng); sự tự cảm và điện cảm; năng lượng của tự điện và cuộn cảm; độ hỗ cảm; thời gian đặc trưng cho mạch $RL$ và $RC$. Mạch xoay chiều: biên độ phức; trở kháng của điện trở, tụ điện và cuộn cảm và của mạch hỗn hợp; sơ đồ pha; công hưởng dòng và thế; công suất tiêu tán.

\subsection{Dao động và sóng}
\subsubsection{Dao động đơn}
Dao động điều hòa: phương trình chuyển động, tần số, tần số góc và chu kỳ. Con lắc vật lý và chiều dài hiệu dụng. Biểu hiện của hệ cân bằng gần không bền. Sự giảm biên độ theo hàm mũ của dao động tắt dần; cộng hưởng của dao động điều hòa cưỡng bức: biên độ và độ lệch pha của dao động bền. Dao động tự do của mạch $LC$; tương đương cơ-điện; phản hồi dương gây ra mất cân bằng; tạo sóng sin bằng phản hồi trên dao động tử $LC$.

\subsubsection{Sóng}
Sự truyền sóng điều hòa: pha dao động là hàm tuyến tính của thời gian và không gian; hàm sóng, vector sóng, vận tốc pha và vận tốc nhóm; sự tắt dần theo hàm mũ của sóng truyền trong môi trường tiêu tán; sóng ngang và sóng dọc; hiệu ứng Doppler cổ điển. Sóng trong môi trường không thuần nhất: Nguyên lý Fermat, định luật Snell. Sóng âm: vận tốc theo hàm của áp suất (module Young hoặc module khối) và mật độ, nón Mach. Năng lượng của sóng: tỉ lệ thuận với bình phương biên độ, sự liên tục của mật độ năng lượng.

\subsubsection{Giao thoa và nhiễu xạ}
Sự chồng chất sóng: nguồn kết hợp, phách, sóng dừng, nguyên lý Huygens, nhiễu xạ bản mỏng (điều kiện cho cường độ cực đại và cực tiểu). Nhiễu xạ 1 khe và 2 khe, cách tử nhiễu xạ, phản xạ Bragg.

\subsubsection{Tương tác của sóng điện từ với môi trường}
Sự phụ thuộc của độ điện thẩm vào tần số (định tính); chiết suất; sự tán sắc và tiêu tán của sóng điện từ trong môi trường trong suốt không màu. Phân cực tuyến tính; góc Brewster; kính phân cực; định luật Malus.

\subsubsection{Quang hình và trắc quang}
Gần đúng của quang hình: tia sáng và ảnh; bóng mờ và bóng sắc nét. Gần đúng cho thấu kính mỏng; dựng ảnh bằng thấu kính mỏng lý tưởng; công thức thấu kính. Quang thông và tính liên tục; độ rọi và cường độ sáng.

\subsubsection{Thiết bị quang học}
Kính hiển vi và kính viễn vọng; kính lúp và năng lực phân giải; cách tử nhiễu xạ và năng lực phân giải; giao thoa kế.

\subsection{Tương đối tính}
Nguyên lý tương đối và phép biến đổi Lorentz cho thời gian và không gian, và cho năng lượng và động lượng; tương đương năng lượng-khối lượng; sự bất biến của khoảng cách không-thời gian và của khối lượng nghỉ. Phép cộng vận tốc song song; dãn thời gian; co ngắn chiều dài; sự tương đối của tính đồng thời; năng lượng và động lượng của photon và hiệu ứng Doppler tương đối tính; công thức chuyển động tương đối tính; bảo toàn năng lượng và động lượng cho va chạm đàn hồi và không đàn hồi của các hạt

\subsection{Vật lý lượng tử}
\subsubsection{Sóng xác suất}
Hạt như là sóng: liên hệ giữa tần số và năng lượng, và giữa vector sóng và động lượng. Mức năng lượng của các hạt nhân giống hydro (chỉ quỹ đạo tròn) và của thế năng parabol; lượng tử hóa moment động lượng. Nguyên lý bất định cho cặp thời gian và năng lượng tương đương, và của tọa độ và động lượng (như định lý, cũng như công cụ ước tính).

\subsubsection{Cấu trúc vật chất}
Phổ hấp thụ và phát xạ của các nguyên tử giống-hydro (chỉ định tính với các nguyên tử khác), và với phân tử do dao động phân tử; độ rộng mức và thời gian sống của các trạng thái kích thích. Nguyên lý loại trừ Pauli cho các hạt Fermion. Hạt (biết điện tích và spin): electron, neutrino electron, proton, neutron, photon; tán xạ Compton. Proton và neutron như là các hạt kết hợp. Hạt nhân, các mức năng lượng của hạt nhân (một cách định tính); phóng xạ alpha, beta và gamma; phân hạch, hợp hạch và bắt giữ neutron, sự giảm khối lượng; thời gian bán phân rã và phân rã theo hàm mũ. Hiện tượng quang điện.

\subsection{Nhiệt động lực học và vật lý thống kê}
\subsubsection{Nhiệt động lực học cổ điển}
Khái niệm về cân bằng nhiệt động và chu trình thuận nghịch; nội năng, công và nhiệt lượng; thang nhiệt độ Kelvin; entropy; hệ đóng, mở, kín; định luật I và II nhiệt động lực học. Thuyết động học phân tử: số Avogadro, hằng số Boltzman và hằng số khí; chuyển động tịnh tiến của phân tử và áp suất; phương trình khí lý tưởng; các bậc tự do tịnh tiến, xoay và dao động; định lý đẳng phân năng lượng; nội năng khí lý tưởng; vận tốc trung bình toàn phương của phân tử. Chu trình đăng nhiệt, đẳng tích, đẳng áp và đoạn nhiệt; nhiệt dung riêng cho chu trình đẳng tích và đẳng áp; chu trình Carnot khí lý tưởng và hiệu suất; hiệu suất động cơ không lý tưởng

\subsubsection{Dẫn nhiệt và chuyển pha}
Chuyển pha (sôi, bốc hơi, nóng chảy, thăng hoa) và ẩn nhiệt; áp suất hơi bão hòa, độ ẩm tương đối; sự sôi; định luật Dalton; khái niệm về truyền nhiệt; sự liên tục của thông lượng nhiệt.

\subsubsection{Vật lý thống kê}
Định luật Planck (giải thích định tính, không nhất thiết phải ghi nhớ), công thức dịch chuyển Wien; định luật Stefan-Boltzmann.

\section{Kỹ năng thực hành}
\subsection{Tổng quan}
Kiến thức lý thuyết để triển khai thí nghiệm phải nằm trong Mục 2 của Danh mục này.\\

Các bài thí nghiệm phải bao gồm một vài ý trong đó quy trình thí nghiệm (triển khai, danh sách các đại lượng cần đo đạc và công thức tính toán) không được mô tả chi tiết.\\

Bài thí nghiệm có thể ngầm chứa các ý lý thuyết (tìm công thức cần thiết để tính toán); không được có ý lý thuyết một cách trắng trợn trừ khi những ý này thử thách sự thông hiểu về nguyên tắc hoạt động của bố trí thí nghiệm hoặc của hiện tượng vật lý cần nghiên cứu, và không được bao gồm tính toán dài dòng.\\

Số lần đo đạc trực tiếp và khối lượng công việc tính toán không được lớn đến mức chiếm phần lớn thời gian được cho: thay vào đó, bài thí nghiệm nên thử thách khả năng sáng tạo, thay vì tốc độ thực hiện tác vụ kỹ thuật của thí sinh.\\

Thí sinh cần những kỹ năng sau đây.\\

\subsection{Bảo hộ}
Biết các quy định an toàn trong khi làm thí nghiệm. Cho dù vậy, nếu quy trình thí nghiệm có khả năng mất an toàn, phải có cảnh báo phù hợp trong bài. Cần tránh các thí nghiệm với nguy cơ mất an toàn lớn.

\subsection{Kỹ năng đo đạc}
Được làm quen với các kỹ năng thí nghiệm thông thường để đo đạc các đại lượng vật lý được nhắc đến ở phần lý thuyết.\\

Biết sử dụng các thiết bị cơ bản tại phòng thí nghiệm và bản tương tự và kỹ thuật số của các thiết bị cơ bản, như thước kẹp, đồng hồ bấm giây, nhiệt kế điện tử, đồng hồ đo điện vạn năng (bao gồm Ohm kế, Volt kế AC/DC và Ampere kế), chiết áp, diode, thấu kính, lăng kính, giá đỡ quang học, Calorie kế và tương tự.\\

Các dụng cụ phức tạp nhiều khả năng không thân thuộc với thí sinh không nên chiếm nhiều phần của bài. Trong trường hợp các thiết bị phức tạp tương đối (dao động ký, máy đếm, máy phát xung, cổng quan điện, v.v), phải có hướng dẫn sử dụng cho thí sinh.\\

\subsection{Độ chính xác}
Nên nhớ rằng thiết bị có thể ảnh hưởng đến kết quả thí nghiệm.\\

Biết các kỹ năng cơ bản để tăng độ chính xác thí nghiệm (v.d đo nhiều khoảng thời gian thay vì một, giảm ảnh hưởng của nhiễu, v.v).\\

Biết được khi cần tìm sự phụ thuộc theo hàm của một đại lượng vật lý, mật độ khảo sát dữ liệu phải phù hợp với đặc tính lân cận của hàm phụ thuộc đó.\\

Biểu diễn kết quả cuối cùng và sai số thực nghiệm với số chữ số hợp lý và làm tròn đúng.\\

\subsection{Phân tích sai số thực nghiệm}
Nhận dạng nguồn sai số chủ đạo, và ước tính hợp lý độ lớn của sai số đo đạc trực tiếp (sử dụng quy tắc từ tài liệu nếu được cho).\\

Phân biệt giữa sai số ngẫu nhiên và sai số hệ thống; biết ước tính và giảm thiểu sai số ngẫu nhiên qua đo đạc nhiều lần.\\

Tìm độ bất định tuyệt đối và tương đối của một đại lượng theo các đại lượng đo đạc được theo các phương pháp hợp lý (xấp xỉ tuyến tính, phép cộng theo module hoặc phép cộng Pythagoras).

\subsection{Phân tích dữ liệu}
Chuyển một liên hệ sang dạng tuyến tính bằng cách chọn các biến phù hợp và áp đường thẳng cho các điểm thực nghiệm. Hồi quy tuyến tính các tham số (gradient, độ lệch và sai số ước tính) hoặc bằng đồ thị, hoặc bằng chức năng thống kê của máy tính cầm tay (cả 2 cách đều được chấp nhận).\footnote{Mặc dù trong bản gốc của danh mục này đã chấp nhận việc sử dụng chức năng thống kê của máy tính cầm tay, song ở một số kỳ thi IPhO, nước chủ nhà đã không tuân theo "danh mục IPhO" và không cho phép thí sinh sử dụng chức năng thống kê này để xử lý số liệu.}\\

Chọn các thang đo phù hợp với đồ thị và biểu diễn các điểm dữ liệu với thanh sai số.\\

\section{Toán bổ trợ}
\subsection{Đại số}
Đơn giản hóa công thức bằng khai triển và nhóm số. Giải hệ phương trình tuyến tính. Giải các phương trình và hệ phương trình dẫn đến phương trình bậc 2 và hệ 2 phương trình bậc 2; sử dụng các lời giải có nghĩa vật lý. Tổng cấp số cộng và cấp số nhân.

\subsection{Hàm}
Các tính chất cơ bản của hàm lượng giác và các hàm nghịch, hàm mũ và logarithm, đa thức. Bao gồm các hằng đẳng thức lượng giác, giải các phương trình đơn giản chứa lượng giác và hàm nghịch, hàm logarithm và hàm mũ.

\subsection{Hình học và hình học không gian}
Độ và radian là các cách đo góc khác nhau. Góc đối đỉnh bằng nhau, góc trong cùng phía và đồng vị bằng nhau. Nhận biết các tam giác đồng dạng. Diện tích tam giác và hình bình hành, hình tròn và ellipse; diện tích mặt cầu, mặt trụ và mặt nón; thể tích khối cầu; khối trụ; hình nón và hình lăng trụ. Quy tắc về sin và cos, góc chắn và góc ở đỉnh cung, định lý Thales, trọng tâm tam giác. Thí sinh cần biết các tính chất của đường conic bao gồm đường tròn, ellipse, parabol và hyperbol.

\subsection{Vector}
Tính chất cơ bản của phép cộng vector, tích vô hướng và tích có hướng. Tính có hướng 2 lần và tích vô hướng 3 lần. Biểu diễn hình học của đạo hàm theo thời gian của một đại lượng vector.\\

\subsection{Số phức}
Phép cộng, nhân, chia số phức; phân biệt phần thực và phần ảo. Chuyển đổi giữa các cách biểu diễn đại số, lượng giác và dạng mũ của một số phức. Nghiệm phức của phương trình bậc 2 và ý nghĩa vật lý (của chúng).

\subsection{Thống kê}
Tính xác suất là tỉ số số lượng của vật thể hoặc tần số diễn ra biến cố. Tính toán giá trị trung bình, độ lệch chuẩn và độ lệch chuẩn trung bình nhóm.

\subsection{Giải tích}
Tìm đạo hàm của các hàm cơ bản, tổng, tích, thương và hàm hợp. Nguyên hàm như là phương thức ngược với đạo hàm. Tìm tích phân xác định và tích phân vô định trong các trường hợp đơn giản: hàm cơ bản, tổng của hàm và dùng phép thế cho các tham số phụ thuộc tuyến tính. Tìm tích phân xác định của đại lượng vô hướng bằng phép thế. Biểu diễn hình học của đạo hàm và tích phân. Tìm hằng số tích phân bằng điều kiện đầu. Khái niệm về gradient của vector (không cần đạo hàm từng phần chặt chẽ).

\subsection{Phương pháp số và xấp xỉ}
Xấp xỉ tuyến tính và đa thức sử dụng chuỗi Taylor. Tuyến tính hóa các biểu thức. Phương pháp nhiễu loạn: tính toán nghiệm chính xác sử dụng các nghiệm thuần nhất đơn giản hơn. Sử dụng phương pháp số xấp xỉ cho các phương trình sử dụng, v.d phương pháp Newton hoặc chia đôi khoảng. Tích phân bằng phương pháp hình bình hành hoặc cộng tam giác.

\end{multicols}
\end{document}
